

The following questions are due to Eric Evert (eric.evert@kuleuven.be) and Bill Helton (helton@math.ucsd.edu).
%\author[E. Evert]{Eric Evert${}^1$}
%\address{Eric Evert, Department of Mathematics\\
%  University of California \\
%  San Diego
%   }
%   \email{eevert@ucsd.edu}
%\thanks{${}^1$Research supported by the NSF grant
%DMS-1500835}
%\author[J.W. Helton]{J. William Helton${}^1$}
%\address{J. William Helton, Department of Mathematics\\
%  University of California \\
%  San Diego
%   }
%   \email{helton@math.ucsd.edu}


\section{Brief definitions}

We begin with basic definitions. Readers familiar with matrix convex sets and their extreme points may proceed directly to Section 
\ref{sec:MatExtreme}.

We warn that these questions and all claims regarding free spectrahedra restrict to the setting of real free spectrahedra. There can be 
important differences for free spectrahedra which contain tuples of complex self-adjoint matrices. 

\subsection{Free spectrahedra and matrix convex sets}

Let $SM_n (\R) ^g$ denote $g$-tuples of symmetric $n \times n$ matrices, and set $SM (\R)^g := \cup_n SM_n (\R)^g$. The following questions 
concern \textit{free spectrahedra}, i.e. solution sets to \textit{linear matrix inequalities}. Given a $g$-tuple $A \in SM_d (\R)^g$, a 
\df{linear matrix inequality} is an inequality of the form 
\[
L_A (x) = I + A_1 x_1 + \dots + A_g x_g \succeq 0.
\]
The map $L_A$ is called a \df{monic linear pencil}. For a matrix tuple $X = (X_1, \dots, X_g) \in SM(\R)^g$, the \df{evaluation} of $L_A$ on 
$X$ is 
\[
L_A (X) = I + A_1 \otimes X_1 + \dots + A_g \otimes X_g. 
\]
Here $\otimes$ denotes the Kronecker product. A \df{free spectrahedron} $\cD_A$ is the solution set 
\[
\cD_A = \{ X \in SM (\R)^g : L_A (X) \succeq 0 \}.
\]
We emphasize that a free spectrahedron contains matrix tuples of all sizes. We let $\cD_A (n)$ denote the set collection of $g$-tuples of $n 
\times n$ matrices in $\cD_A$. We say a free spectrahedron is \df{compact} if $\cD_A (n)$ is compact for each $n$.

Free spectrahedra are prototypical examples of \textit{matrix convex sets}. A \df{matrix convex combination} of a collection of tuples 
$\{Y^i\} \in SM(\R)^g$ is a finite sum of the form 
\beq
\label{eq:SumY}
\sum_{i=1}^k V_i^T Y^i V_i \qquad \mathrm{where} \qquad \sum_{i=1}^k V_i^T V_i = I_n.
\eeq
Here each $Y^i=(Y_1^i, \dots, Y_g^i)$ is a $g$-tuple of $n_i \times n_i$ matrices and the $V_i$ are $n_i \times n$ contractions. The product 
$V_i^T Y^i V_i$ is defined by 
\[
V_i^T Y^i V_i = (V_i^T Y_1^i V_i, \dots, V_i^T Y_1^i V_i).
\]
A matrix convex combination is said to be \df{proper} if each $V_i$ is surjective or \df{weakly proper} if $V_i \neq 0$ for each $i$. 

An important aspect of matrix convex combinations is that the $n_i$ need not be equal. I.e. the $Y^i$ may be tuples of matrices of different 
sizes. A set is called \df{matrix convex} if it is closed under matrix convex combinations. All free spectrahedra are matrix convex and every 
matrix convex set can be represented as a (perhaps infinite) intersection of free spectrahedra. 

A few warm-up exercises which are good for those interested in studying free spectrahedra and matrix convex sets are
\begin{enumerate}
\item Show every free spectrahedron is matrix convex.
\item Show that if $K$ is a matrix convex set, then $K(n)$ is convex in the classical sense.
\item Show that if $K$ is a matrix convex set and $K$ is nonempty, then $K$ contains matrix tuples of all sizes. In other words, show that if 
$K$ is matrix convex and there exists a tuple $X \in K(m)$ for some integer $m$, then for any integer $n$ there exists a tuple $Y \in K(m)$. 
\end{enumerate}

A (very) brief selection of recommended articles to read for basic information on Free spectrahedra and matrix convex sets is 
\cite{EW97,HM12,HKM13}.

\subsection{Extreme points of free spectrahedra}

There are several notions of an extreme point for a matrix convex set. The main types we consider are \textit{Euclidean (classical) extreme 
points, matrix extreme points}, and \textit{free extreme points}.



Suppose $\cD_A$ is a free spectrahedron. We say $X \in \cD_A (n)$ is a Euclidean extreme point of $\cD_A$ if $X$ is an extreme point of 
$\cD_A$ in the classical sense. 

A tuple  $X \in \cD_A(n)$  is a \df{matrix extreme point}
  of $\cD_A$
 if whenever it is represented as a proper matrix combination of the form
\eqref{eq:SumY} with $Y^i \in \cD_A(n_i)$,
then $n=n_i$ and for each $i$ there is a unitary $U_i$ so that  $X = U_i^T Y^i U_i$.

 A tuple  $X \in \cD_A(n)$  is an \df{free extreme point}   of $\cD_A$
 if whenever it is represented as a weakly proper matrix combination of the form \eqref{eq:SumY},  then
  for each $j$ either $n_j \leq n$ and there is a unitary $U_j$ so that  $X = U_j^T Y^j U_j$ (and hence $n_j=n$),
 or $n_j>n$ and there exists a $Z^j  \in \cD_A$ and a unitary $U_j$  such that $U_j^T Y^j U_j =  X\oplus Z^j$.
 
 In words, a tuple $X$ is a matrix extreme point a free spectrahedron if $X$ cannot be expressed as a nontrivial matrix convex combination of 
 elements of $\cD_A$ having size less than or equal to $X$, while a $X$ is a free extreme point if $X$ cannot be expressed as any nontrivial 
 convex combination of elements of the free spectrahedron. 
 
 We have the following relationship between extreme points of free spectrahedra.
 \begin{theorem}
 Let $\cD_A$ be a free spectrahedron. If $X$ is a free extreme point of $\cD_A$, then $X$ is a matrix extreme point of $\cD_A$. In addition, 
 if $X$ is a matrix extreme point of $\cD_A$, then $X$ is Euclidean extreme point of $\cD_A$. 
 \end{theorem}
 
 A (very) brief selection of recommend articles to read for an introduction to extreme points of free spectrahedra and matrix convex sets is 
 \cite{WW99,EHKM18,EH19}.
 
It is worth mentioning that free extreme points can be seen as a finite dimensional version of the classical dilation theoretic Arveson 
boundary. We will not discuss this perspective further; however we recommend \cite{A69,DK15,DK+} as a starting ponit for the interested 
reader. Also see \cite{F00,F04} for discussions of matrix extreme points from this perspective.
 
 We now present a list of questions related to free spectrahedra and their extreme points.

\section{Are matrix extreme points equal to free extreme points in a free spectrahedron}
\label{sec:MatExtreme}

While it is known that the set of matrix extreme points can be a proper subset of the Euclidean extreme points of a free spectrahedron, there 
are no known examples of a matrix extreme point which is not a free extreme point of a free spectrahedron.

\begin{question}
\label{question:MissingMatExtreme}
Let $\cD_A$ be a free spectrahedron with $A$ a tuple of real symmetric matrices. Is the set of matrix extreme points of $\cD_A$ equal to the 
set of free extreme points of $\cD_A$. 
\end{question}

For general matrix convex sets there are examples of matrix extreme points which are not free extreme points. More dramatically, there are 
examples of matrix convex sets which are not free spectrahedra which have no free extreme points, e.g. see \cite{E18}. On the other hand, 
\cite{WW99} shows that all (compact) matrix convex sets are spanned by matrix extreme points. It is known that all free spectrahedra are 
spanned by free extreme points, see \cite[Theorem 1.1]{EH19}.

\section{Noncommutative varieties}

A small number of free spectrahedra are known to have the property that their free extreme points are determined by a noncommutative variety. 
Defined by example, a \df{noncommutative (NC) polynomial} is a polynomial in the noncommuting indeterminates $x=(x_1,\dots,x_g)$, e.g. 
\[
p(x) = x_1 x_2 + 2 x_3 x_1 x_2 x_1
\]
is a NC polynomial. \df{Evaluation} of a NC polynomial on a $g$-tuple of matrices is defined by replacing $x_i \to X_i$, e.g. 
\[
p(X) = X_1 X_2 + 2 X_3 X_1 X_2 X_1. 
\]
A noncommutative variety is the zero set of a finite collection of NC polynomials. That is, given a finite collection $\{p_i\}_{i=1}^k$ of NC 
polynomials an \df{noncommutative variety}\footnote{This terminology varies from paper to paper.} is a set of the form 
\[
\{ X \in SM (\R)^g : p_i(X) = 0 \ \mathrm{for \ all \ } i=1,\dots,k\}.
\]

\begin{question}
Which free spectrahedra have the property that the set of free extreme points of the free spectrahedron is given by a noncommutative variety? 
Are there examples of free spectrahedra which do not have this property.
\end{question}

The free spectrahedra known to have free extreme points determined by a noncommutative variety are
\begin{enumerate}
\item The free cube in $g$-variables. I.e. the free spectrahedron $\mathcal{C}^g$ defined by
\[
\mathcal{C}^g = \{X \in SM(\R)^g : X_i^2 \preceq I \ \mathrm{for \ all} \ i = 1,\dots, g\}.
\]
\item Free simplices. I.e. free spectrahedra in $g$ variables defined by a tuple of $g+1 \times g+1$ diagonal matrices.
\item Free quadrilaterals. I.e. free spectrahedra in $2$ variables defined by a tuple of $4 \times 4$ diagonal matrices.
\item The spin ball. I.e. the free spectrahedron with defining pencil $L_A$ where 
\[
A = \left(\begin{pmatrix}
1 & 0 \\ 
0 & -1
\end{pmatrix},
\begin{pmatrix}
0 & 1 \\ 
1 & 0
\end{pmatrix}
\right)
\]
\end{enumerate}

Limited numerical evidence suggests that free spectrahedra with free extreme points determined by a noncommutative variety are rare. See 
\cite{E+} for details.



\section{Free spectrahedrops}

We now give various questions related to free spectrahedrops, i.e. projections of free spectrahedra; some call these spectrahedral shadows. 
Let $\cD_{(A,\tilde{A})}$ be a free spectrahedron with elements $(X,Y)$. That is $\cD_{(A,\tilde{A})}$ is the set of solutions to the linear 
matrix inequality
\[
I+A_1 \otimes X_1 + \cdots + A_g \otimes X_g + \tilde{A}_1 \otimes \tilde{Y}_1 + \cdots \tilde{A}_{\tilde{g}} \otimes \tilde{Y}_{\tilde{g}} 
\succeq 0. 
\]
We define the \df{free spectrahedrop} $P_X \cD_{(A,\tilde{A})}$ by
\[
P_X \cD_{(A,\tilde{A})} = \{ X \in SM(\R)^g : \mathrm{\ There \ exists\ a \ } Y \in SM(\R)^{\tilde{g}} \ \mathrm{such\ that\ } (X,Y) \in 
\cD_{(A,\tilde{A})} \}
\]
See \cite{HKMjems} for details about free spectrahedrops. 
 
\subsection{Extreme points of free spectrahedrops}

\begin{question}
Let $P_X \cD_{(A,\tilde{A})}$ be a free spectrahedrop which is closed under complex conjugation. Is the set of free extreme points of $P_X 
\cD_{(A,\tilde{A})}$ nonempty? Moreover, is $P_X \cD_{(A,\tilde{A})}$ equal to the matrix convex hull of its free extreme points?
\end{question}

\cite{EHKM18} shows that the set of free extreme points of a free spectrahedron is nonempty, and \cite{EH19} shows that every free 
spectrahedron is the matrix convex hull of its free extreme points. 

\begin{question}
Let $K$ be a free spectrahedrop, and let $\cD_{(A,\tilde{A})}$ be a free spectrahedron such that $P_X \cD_A = K$. Suppose $Y$ is a free 
extreme point of $K$. Is there always a tuple $\tilde{Y}$ such that $(Y,\tilde{Y})$ is a free extreme point of $\cD_A$. 
\end{question}

Note: A negative answer to the above question would resolve Question \ref{question:MissingMatExtreme}. Using the fact that a matrix convex 
set at level $n$ is spanned by its matrix extreme points at level $n$, one show that there is always a $\tilde{Y}$ such that $(Y,\tilde{Y})$ 
is matrix extreme.

\subsubsection{Drops and hulls}

\begin{question}
Characterize which free matrix convex sets are free spectrahedrops. More generally, characterize free sets which are projections of free 
semialgebraic sets: This is one of the most basic problems in free real algebraic geometric.
\end{question}

\cite{HM12} shows that if $K$ is free semialgebraic set which is level wise convex set and has $0$ in its interior, then $K$ is a free 
spectrahedron. It would be very interesting to have some kind of characterization of free spectrahedrops. 

Before stating the next question we introduce a small amount of notation. Given a NC polynomial $p$ with $p(0)=1$ we let $\cD_p$ denote the 
connected component around $0$ of $\{X : p (X) \succeq 0\}$. 

\begin{question}
For $p(x) = 1-x_1^2-x_2^4$ is the matrix convex hull of $\cD_p$ a free spectrahedrop. In general, for which NC polynomials $p$ is the matrix 
convex hull of $\cD_p$ a free spectrahedrop. 
\end{question}

$\cD_p$ is often called the bent TV screen. See \cite[Section 7.3]{EHKM18} for an in depth discussion of this domain. The only known examples 
for the above question are sets which are either convex or ``secretly" convex, see \cite{HKM16}.



\section{Free polar duals}

Given a matrix convex set $K \subset SM(\R)^g$, the \df{polar dual} of $K$, denoted $K^\circ$ is the set
\[
K^\circ = \{ Y \in SM(\R)^g : L_X (Y) \succeq 0 \mathrm{\ for\ all\ } X \in K\}. 
\]

\begin{question}
Is there a free spectrahedron $\cD_A$ such that the polar dual of $\cD_A$ is also a free spectrahedron and such that $\cD_A$ is not a free 
simplex. Equivalently, if $\cD_A$ is a free spectrahedron which is not a simplex, does $\cD_A$ necessarily have infinitely many (unitary 
equivalence classes of) free extreme points.
\end{question}

\cite[Corollary 6.8]{EHKM18} and \cite[Theorem 4.7]{FNT17} independently show that the polar dual of a free simplex is again a free simplex. 
In \cite{EHKM18} it is shown that the polar dual of a matrix convex set is a free spectrahedron if and only if the matrix convex set has 
finitely many (unitary equivalence classes of) free extreme points. It is not known if there are free spectrahedra which are not free 
simplices that have finitely many free extreme points. 


\begin{thebibliography}{1}
	\bibitem[A69]{A69} W. Arveson:
	{\it Subalgebras of $C^*$-algebras}, Acta Math. {\bf 123} (1969) 141-224.
		
	
	\bibitem[DK15]{DK15} K.R. Davidson, M. Kennedy:
	{\it The Choquet boundary of an operator system}, Duke Math. J. {\bf 164} (2015) 2989-3004.
	
	\bibitem[DK+]{DK+}
	K.R. Davidson, M. Kennedy:
	\textit{Noncommutative Choquet Theory},
	preprint \url{https://arxiv.org/pdf/1905.08436.pdf}.
	
	
	\bibitem[EW97]{EW97} E.G. Effros, S. Winkler: {\it Matrix convexity: operator analogues of the bipolar and Hahn-Banach theorems}, J. 
	Funct. Anal. {\bf 144} (1997) 117-152.
	
	\bibitem[E18]{E18} E. Evert: {\it Matrix convex sets without absolute extreme points}, Linear Algebra Appl. {\bf 537} (2018) 
	287-301.		
	\bibitem[E+]{E+}
	E. Evert: {\it The Arveson boundary of a Free Quadrilateral is given by a noncommutative variety}, preprint
		https://arxiv.org/abs/2008.13250
	
	\bibitem[EH19]{EH19} E. Evert, J.W. Helton: {\it Arveson extreme points span free spectrahedra},	Math. Ann. {\bf 375} (2019) 
	629-653.
	
	\bibitem[EHKM18]{EHKM18}
	E. Evert, J.W. Helton, I. Klep, S. McCullough:
	{\it Extreme points of matrix convex sets, free spectrahedra and dilation theory},
	J. of Geom. Anal. {\bf 28} (2018) 1373-1498.
	
	\bibitem[EOYH19]{EOYH19} E. Evert, M. de Oliveira, J. Yin, and J.W. Helton: {\it NCSE 1.0: An NCAlgebra for optimization over free 
	spectrahedra}, Available online, Jan. 2019. URL:
	https://github.com/NCAlgebra/UserNCNotebooks/tree/master/NCSpectrahedronExtreme
	
	\bibitem[F00]{F00} D.R. Farenick:
	{\it Extremal matrix states on operator systems}, J. London Math. Soc. {\bf 61} (2000) 885-892.
	
	\bibitem[F04]{F04} D.R. Farenick:
	{\it Pure matrix states on operator systems}, Linear Algebra Appl. {\bf 393} (2004) 149-173.
	
		
	\bibitem[FNT17]{FNT17} T. Fritz, T. Netzer, A. Thom: {\it Spectrahedral Containment and Operator Systems with Finite-dimensional 
	Realization}, SIAM J. Appl. Algebra Geom. {\bf 1} (2017) 556-574.
	
	
	\bibitem[HKM13]{HKM13} J.W. Helton, I. Klep, S. McCullough: {\it The matricial relaxation of a linear matrix inequality}, Math. 
	Program. {\bf 138} (2013) 401-445.
	
	\bibitem[HKM16]{HKM16}
	J.W. Helton, I. Klep, S. McCullough:
	{\it Matrix convex hulls of free semialgebraic sets}, Trans. Amer. Math. Soc. {\bf 368} (2016) 3105--3139.
	
	
	\bibitem[HKM17]{HKMjems}
	J.W. Helton, I. Klep, S. McCullough:
	{\it The tracial Hahn-Banach theorem, polar duals, matrix convex sets, and projections of free spectrahedra}, J. Eur. Math. Soc. {\bf 6} 
	(2017) 1845--1897.
	
	
	\bibitem[HM12]{HM12}
	J.W. Helton, S. McCullough: {\it Every free basic convex semi-algebraic set has an LMI representation}, Ann. of Math. (2) {\bf 176} 
	(2012) 979-1013.
	
	
	\bibitem[WW99]{WW99}
	C. Webster and S. Winkler:
	{\it The Krein-Milman Theorem in Operator Convexity},  Trans Amer. Math. Soc. {\bf 351} (1999) 307-322.

\end{thebibliography}
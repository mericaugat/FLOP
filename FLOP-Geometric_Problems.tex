


\section{Injektivit{\"a}tstellensatz?}

Suppose $\bbp\in (\C\fralg{\bbx})^\tth$, that is $\bbp = (p_1,p_2,\dots, p_\tth)$.
If $q$ is a free polynomial that is in the unital algebra generated by $\bbp$, then there exists some free polynomial $f$ (in 
$\tth$-variables) such that $q = f\circ \bbp$.
It is quickly seen that in this case $q$ inherits the injectivity failings of $\bbp$.
That is, $q(X) = q(Y)$ whenever $\bbp(X) = \bbp(Y)$.

The question is whether the above condition is a certificate of inclusion in the algebra generated by $\bbp$.

\begin{problem}
	Suppose $\bbp\in (\C\fralg{\bbx})^\tth$ and $q\in \C\fralg{\bbx}$.
	If $q(X) = q(Y)$ whenever $\bbp(X) = \bbp(Y)$ (for $X,Y\in M_n(\C)^\ttg$ or perhaps a different free set), then is it the case that $q$ 
	is in the unital algebra generated by $\bbp$?
\end{problem}


Since we are examining the unital algebra, it is safe to assume that $\bbp(0) = 0$ and $q(0) = 0$.
With this additional assumption, we have in particular that $q(X) = 0$ whenever $\bbp(X) = 0$.
Hence, $q$ is in the ideal generated by $\bbp$ (however this is naturally the case when $q$ is in the unital algebra generated by $\bbp$ and 
they both send $0$ to $0$).
However, membership in the ideal is not enough to guarantee that a polynomial shares the same lack of injectivity.
For example, $x^3$ is in the ideal generated by $x^2$, but $(-1)^2 = 1 = 1^2$ while $(-1)^3 \neq (1)^3$.


We have the following result for free functions that gives a connection between our problem's conjectured certificate and zero sets of 
derivatives.

\begin{proposition}
	Suppose $\bm{F}$ is a free mapping (a $d$-tuple of free maps) on a free domain $\Omega$ and $G$ is a free map.
	Then $G(X) = G(Y)$ whenever $\bm{F}(X) = \bm{F}(Y)$ (for $X,Y\in \Omega$) if and only if $DG(Z)[W] = 0$ whenever 
	$D\bm{F}(Z)[W] = 0$ (for $Z\in \Omega$ and $W\in M(\C)^\ttg$).
\end{proposition}

Thus, some mileage may be gained from investigating the zero sets of derivatives.


\section{Polynomial Automorphisms between Free Loci}

Suppose $A = (A_1,\dots, A_\ttg)\in M_k(\C)^\ttg$ and let $L_A(x) = I_k - \sum_{i=1}^\ttg A_ix_i$.
The function $L_A$ is a monic linear pencil and it can be evaluated on $\ttg$-tuples of matrices in the natural way.
If $X = (X_1,\dots, X_\ttg)\in M_n(\C)^\ttg$ then
\[
	L_A(X) = I_k\otimes I_n - \sum_{i=1}^\ttg A_i\otimes X_i
\]
where $\otimes$ is the Kronecker product. For each $n\in \Z^+$ define
\[
	\mathscr{Z}_n(L_A) = \set{X\in M_n(\C)^\ttg \, : \, \det(L_A(X)) = 0}
\]
and let
\[
	\mathscr{Z}_n(L_A) = \bigsqcup_{n=1}^\infty \mathscr{Z}_n(L_A).
\]
That is, $\mathscr{Z}(L_A)$ is the free set whose levels are $\mathscr{Z}_n(L_A)$ and we call $\mathscr{Z}(L_A)$ the \df{free locus} of $L_A$.
There are many significant results involving free loci \cite{KlVol17}.

\begin{problem}
	Suppose $A\in M_n(\C)^\ttg$ and $B\in M_m(\C)^\ttg$. Describe the polynomial automorphisms between $\mathscr{Z}(L_A)$ and 
	$\mathscr{Z}(L_B)$. When non-trivial automorphisms exist, how are $A$ and $B$ related? If needed, assume $n=m$.
\end{problem}




\begin{thebibliography}{1}

\bibitem[KlVol17]{KlVol17}
Klep, I. and Vol{\v c}i{\v c}, J.
\newblock Free loci of matrix pencils and domains of noncommutative rational functions.
\newblock {\em Commentarii Mathematici Helvetici}, 92(1):105--130, 2017.



\end{thebibliography}
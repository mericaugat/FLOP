
\begin{problem}
	Suppose $A\in M_n(\C\fralg{\bbx}\otimes \C\fralg{\bby})$. If $A$ is invertible over the $\C$-algebra $\C\fralg{\bbx}\otimes \C\fralg{\bby}$ then do there exist $D,E_1,\dots, E_k\in M_n(\C\fralg{\bbx}\otimes \C\fralg{\bby})$ such that $D$ is diagonal, each $E_i$ is an elementary matrix and $A = DE_1\dots E_k$?
\end{problem}

\begin{remark}
	The above problem is false when $n=2$. Cohn gave the following matrix that is not a product of elementary matrices over $\C[x]\otimes \C[y]$:
	\[
		\bpm 1\otimes 1 + x\otimes y & 1\otimes y^2 \\ -x^2\otimes 1 & 1\otimes 1 - x\otimes y \epm.
	\]
\end{remark}

The above counterexample is nonexistent when our matrices are over $\C\fralg{\bbx}$ instead. The proof of this uses results from \cite{SuslinCohn06}.

\begin{theorem}
	If $A\in M_n(\C\fralg{\bbx})$ is invertible, then there exist $D,E_1,\dots, E_k\in M_n(\C\fralg{\bbx})$ such that $D$ is diagonal, each $E_i$ is an elementary matrix and $A = DE_1\dots E_k$.
\end{theorem}

The Cohn counterexample is essentially the only issue:

\begin{theorem}[Suslin's Stability Theorem]
	Suppose $A\in M_n(\C[t_1,\dots, t_\ttg])$ where $n\geq 3$. If $\det(A) = 1$ then there exist $E_1,\dots, E_k\in M_n(\C[t_1,\dots, t_\ttg])$ such that $A = E_1\dots E_k$.
\end{theorem}

An ``algorithmic" proof of the above theorem can be found in \cite{SuslinPW95}.


\begin{remark}
	An idea related to this is the notion of Jacobian Tame, see \ref{sec:Jac Tame}.
\end{remark}


\begingroup
\renewcommand{\addcontentsline}[3]{}% Remove functionality of \addcontentsline
\renewcommand{\section}[2]{}% Remove functionality of \section

\vspace{1em}

\begin{center}
	{\normalsize \textsc{References}}
\end{center}


\begin{thebibliography}{PW95}

\bibitem[Coh06]{SuslinCohn06}
P.M. Cohn.
\newblock {\em Free Ideal Ring and Localizations in General Rings}.
\newblock Cambridge University Press, 2006.

\bibitem[PW95]{SuslinPW95}
H.J. Park and C.~Woodburn.
\newblock An algorithmic proof of {S}uslin's stability theorem for polynomial
  rings.
\newblock {\em Journal of Algebra}, 178(1):277 -- 298, 1995.

%\bibitem[Swe70]{Swe70}
%Moss~Eisenberg Sweedler.
%\newblock A units theorem applied to hopf algebras and amitsur cohomology.
%\newblock {\em American Journal of Mathematics}, 92(1):259--271, 1970.

\end{thebibliography}


\endgroup





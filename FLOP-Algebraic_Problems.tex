




\section{Algebra Notation}

\begin{center}
\begin{tabular}{l c l}
	$\C\fralg{x_1,\dots,x_\ttg}$ &  & free $\C$-algebra generated by $x_1,\dots, x_{\ttg}$ \\
	$\C\skf{x_1,\dots,x_\ttg}$ &  & free skew field (nc rationals) of $\C\fralg{x_1,\dots,x_\ttg}$  \\
	$\C\skf{\bbx}_0$ &  & algebra of rationals regular at $0$ \\
	$\C\skf{\bbx \lra \bby}$ &  & universal skew field of fractions of $\C\skf{\bbx}\otimes \C\skf{\bby}$
\end{tabular}
\end{center}










%############################################################################
%############################################################################
%############################################################################

\section{Factoring Invertible Matrices over $\C\fralg{x}\otimes \C\fralg{y}$}
	\label{sec:Elem_Mats}

\begin{problem}
	Suppose $A\in M_n(\C\fralg{\bbx}\otimes \C\fralg{\bby})$. If $A$ is invertible over the $\C$-algebra $\C\fralg{\bbx}\otimes 
	\C\fralg{\bby}$ then do there exist $D\in M_n(\C)$ and $E_1,\dots, E_k\in M_n(\C\fralg{\bbx}\otimes \C\fralg{\bby})$ such that $D$ is 
	diagonal, each $E_i$ is an elementary matrix and $A = DE_1\dots E_k$?
\end{problem}

\begin{remark}
	This problem is false when $n=2$. Cohn gave the following matrix that is not a product of elementary matrices over $\C[x]\otimes \C[y]$:
	\[
		\bpm 1\otimes 1 + x\otimes y & 1\otimes y^2 \\ -x^2\otimes 1 & 1\otimes 1 - x\otimes y \epm.
	\]
\end{remark}

The above counterexample is nonexistent when our matrices are over $\C\fralg{\bbx}$ instead of $\C\fralg{\bbx}\otimes \C\fralg{\bby}$. The 
proof of this uses results from \cite{Cohn06}.

\begin{theorem}
	If $A\in M_n(\C\fralg{\bbx})$ is invertible, then there exist $D\in M_n(\C)$ and $E_1,\dots, E_k\in M_n(\C\fralg{\bbx})$ such that $D$ is 
	diagonal, each $E_i$ is an elementary matrix and $A = DE_1\dots E_k$.
\end{theorem}

The Cohn counterexample is essentially the only issue:

\begin{theorem}[Suslin's Stability Theorem]
	Suppose $A\in M_n(\C[t_1,\dots, t_\ttg])$ where $n\geq 3$. If $\det(A) = 1$ then there exist $E_1,\dots, E_k\in M_n(\C[t_1,\dots, 
	t_\ttg])$ such that $A = E_1\dots E_k$.
\end{theorem}

An ``algorithmic" proof of the above theorem can be found in \cite{PW95}.


\begin{remark}
	An idea related to this is the notion of Jacobian Tame, see \ref{sec:Jac Tame}.
\end{remark}










%############################################################################
%############################################################################
%############################################################################




\section{Free L{\"u}roth Theorem}
	\label{sec:Luroth}
	
\begin{problem}[Free L{\"u}roth Theorem]
Suppose $\k$ is an algebraically closed infinite field and $x_1,\dots, x_g$ are freely noncommuting indeterminates ($g\geq 2$).
If $\k\subsetneq D\subsetneq \k\skf{x_1,\dots, x_\ttg}$ is a subfield, then do there exist $q_1,\dots, q_\tth\in \C\skf{x_1,\dots, x_\ttg}$ 
such that $D = \C\skf{q_1, \dots, q_\tth}$?

In other words, is every non-trivial subfield of $\C\skf{\bbx}$ a free skew field?
\end{problem}


\begin{theorem}[L{\"u}roth's Theorem]

Suppose $\k$ is a field. If $\k \subsetneq D \subsetneq \k(t)$ is a subfield, then there exists $q\in \k(t)$ such that $D = \k(q(t))$.

\end{theorem}

\begin{remark}
	If $\k$ is algebraically closed and infinite, then L{\"u}roth's Theorem holds as well for $\k(t_1,t_2)$.
	
	On the other hand, there are counterexamples to L{\"u}roth's Theorem for $\k(t_1,t_2,t_3)$.
\end{remark}

\begin{remark}
	Schofield \cite{Sch85} shows that if $f,g\in \C\skf{\bbx}$, then either $[f,g] = 0$ or $\C\skf{f,g}$ is free.
\end{remark}














%############################################################################
%############################################################################
%############################################################################



\section{Free Bertini Theorem - \textit{SOLVED!}}
	\label{sec:FreeBertini}

\noindent Great result solved in \cite{Vol19}


We say $f\in \bbk\fralg{\bbx}$ \textbf{factors} if $f = f_1f_2$ for some nonconstant $f_1,f_2\in \bbk\fralg{\bbx}$. Otherwise $f$ is 
\textbf{irreducible} over $\bbk$.
A nonconstant $f\in \bbk\fralg{\bbx}$ is \textbf{composite} if there exist $h\in \bbk\fralg{\bbx}$ and a univariate polynomial $p\in \bbk[t]$ 
such that $\deg(p)>1$ and $f = p\circ h = p(h)$.


\begin{theorem}
	If $f\in \bbk\fralg{\bbx}\setminus \bbk$ is not composite, then $f - \lambda$ is irreducible over $\overline{\bbk}$ for all but finitely 
	many $\lambda\in\overline{\bbk}$.
\end{theorem}














%############################################################################
%############################################################################
%############################################################################

\section{Rational Bertini Theorem}
	\label{sec:RatBertini}
	
	
	
A nonconstant $f\in \bbk\skf{\bbx}$ is \textbf{composite} if there exist $h\in \bbk\skf{\bbx}$ and a univariate rational $p\in \bbk(t)$ such 
that $p$ is not M{\"o}bius and $f = p\circ h = p(h)$.

\begin{problem}
	A rational function $r$ is not composite if and only if the hypersurfaces
	\[
		\set{X\in \dom_n(r) \, : \, \det(r(X) - \lambda I) = 0} \subset M_n(\C)^\ttg
	\]
	are irreducible for all but finitely many $n\in \N$ and $\lambda\in \C$.
\end{problem}











%############################################################################
%############################################################################
%############################################################################

\section{Jacobian Tame}
	\label{sec:Jac Tame}

An automorphism $\tau$ of the free algebra $\C\fralg{x_1,\dots, x_\ttg}$ is {\bf elementary} if $\tau:x_i \mapsto x_i$ 
	for $i\neq j$ and $\tau:x_j \mapsto cx_j + f$, where $c\in \C$ and $f\in \C\fralg{x_1,\dots, x_{j-1}, x_{j+1},\dots, x_\ttg}$.
We say an automorphism is {\bf tame} if it is a composition of elementary automorphisms.
If the automorphism is not tame, then it is {\bf wild}.

In \cite{U07} it is shown that the Anick automorphism, 
\[
	\delta(x,y,z) = (x + z(xz - zy), y + (xz - zy)z, z)\in \C\fralg{x,y,z}
\]
is wild.
Looking at it, the Anick automorphism doesn't seem particularly ``wild," so maybe this can be improved?
Perhaps it is something about Jacobian matrices?

\begin{definition}
	We say an automorphism $\tau$ of $\C\fralg{x_1,\dots, x_\ttg}$ is \textbf{Jacobian Tame} if
	$J_\tau = DE_1\dots E_k$, where $D\in M_\ttg(\C\fralg{\bbx'}^{opp}\otimes \C\fralg{\bbx})$ is diagonal and
	$E_1,\dots, E_k\in M_\ttg(\C\fralg{\bbx'}^{opp}\otimes \C\fralg{\bbx})$ are elementary matrices.
	
	If such a factorization does not exist, then we say $\tau$ is \textbf{Jacobian wild}.
\end{definition}


\begin{problem}
	Are there any Jacobian wild automorphisms of the free algebra?
	The only known (to me) example of a wild automorphism of $\C\fralg{x,y,z}$ is Jacobian tame (this is explained below).
	
	In the commutative case there are no Jacobian wild automorphisms.
	Every automorphism of $\C[t_1,t_2]$ is tame, hence Jacobian tame.
	If $\phi$ is an automorphism of $\C[t_1,\dots, t_\ttg]$ with $\ttg>2$, then its Jacobian matrix
	$J_\phi\in M_\ttg(\C[t_1,\dots, t_\ttg,t_1',\dots,t_\ttg'])$ factors into such a product by Suslin's Stability Theorem (see 
	\ref{sec:Elem_Mats}), thus is Jacobian tame.
\end{problem}

I will be using the transposed Jacobian matrix: $J_\tau \in M_\ttg(\C\fralg{\bbx'}^{opp}\otimes \C\fralg{\bbx})$ where the $i^\text{th}$ 
column of $J_\tau$ corresponds to the derivatives of $\tau_i$.

It turns out that the Anick automorphism 
\[
	\delta(x,y,z) = (x + z(xz - zy), y + (xz - zy)z, z)\in \C\fralg{x,y,z}
\]
has a Jacobian matrix that can be written as a product of elementary matrices.
The Jacobian matrix of $\delta$ is
\[
	J_\delta = 
	\bpm
		1\otimes 1 + z'\otimes z & 1\otimes z^2 & 0 \\
		-(z')^2\otimes 1 & 1 - z'\otimes z & 0 \\
		\zeta_1 & \zeta_2 & 1
	\epm
\]
where $\zeta_1 = 1\otimes xz+ x'z'\otimes 1 - 1\otimes zy - z'\otimes y$ and 
	$\zeta_2 = x\otimes z + z'x'\otimes 1 - 1\otimes yz - y'z'\otimes 1$.
Let $E_{i,j}(\alpha) = I + \alpha e_{i,j}$ ($e_{i,j}$ has a $1$ in the $i,j$ entry and 0's elsewhere) and observe
\[
	J_\delta E_{3,1}(-\zeta_1)E_{3,2}(-\zeta_2) = 
	\bpm
		1\otimes 1 + z'\otimes z & 1\otimes z^2 & 0 \\
		-(z')^2\otimes 1 & 1 - z'\otimes z & 0 \\
		0 & 0 & 1
	\epm.
\]
In 1966, Cohn proved that
\[
	\mathfrak{C} = 
	\bpm
		1\otimes 1 + z'\otimes z & 1\otimes z^2\\
		-(z')^2\otimes 1 & 1 - z'\otimes z
	\epm
\]
cannot be written as a product of elementary matrices over $\mathbb{F}[z'\otimes 1,1\otimes z]$.
This is a crucial aspect of Umirbaev's (\cite{U07}) proof that the Anick automorphism is wild.
However, Park and Woodburn \cite{PW95} give a decomposition of $(\begin{smallmatrix} \mathfrak{C} & 0 \\ 0 & 1 \end{smallmatrix})$ 
	into a product of elementary matrices:
\begin{align*}
	J_\delta &E_{3,1}(-\zeta_1)E_{3,2}(-\zeta_2) =
	\bpm
		1\otimes 1 + z'\otimes z & 1\otimes z^2 & 0 \\
		-(z')^2\otimes 1 & 1 - z'\otimes z & 0 \\
		0 & 0 & 1
	\epm\\
	= &	
	E_{2,3}(-z'\otimes 1)E_{1,3}(1\otimes z)E_{3,2}(1\otimes z)E_{3,1}(z'\otimes 1)\\
	& E_{2,3}(z'\otimes 1)E_{1,3}(-1\otimes z)E_{3,2}(-1\otimes z)E_{3,1}(-z'\otimes 1).
\end{align*}
Thus
\begin{align*}
	J_\delta = & E_{2,3}(-z'\otimes 1)E_{1,3}(1\otimes z)E_{3,2}(1\otimes z)E_{3,1}(z'\otimes 1)\\
	&E_{2,3}(z'\otimes 1)E_{1,3}(-1\otimes z)E_{3,2}(-1\otimes z)E_{3,1}(-z'\otimes 1)E_{3,2}(\zeta_2)E_{3,1}(\zeta_1).
\end{align*}
Thus, the Jacobian matrix of $\delta$ factors as a product of elementary matrices even though it is wild.


\vspace{1em}



Naturally, a positive answer to the Problem in \ref{sec:Elem_Mats} would show that every automorphism is Jacobian Tame.
On the other hand, a negative answer to the Problem in \ref{sec:Elem_Mats} would only serve to complicate things since a Jacobian matrix will 
certainly look quite different from a generic matrix in $\GL_\ttg(\C\fralg{\bbx'}^{opp}\otimes \C\fralg{\bbx})$.















%############################################################################
%############################################################################
%############################################################################

\section{Tensor Product of Free Skew Fields}
	\label{sec:TPFSF}
	

\begin{problem}
	Classify the invertible elements of $\C\skf{\bbx}\otimes \C\skf{\bby}$.
	The expectation is that if $Q$ and $Q^{-1}$ are in $\C\skf{\bbx}\otimes \C\skf{\bby}$, then $Q = r\otimes s$, for some nonzero elements 
	$r\in \C\skf{\bbx}$ and $s\in \C\skf{\bby}$.
\end{problem}

The following was proven in \cite{Swe70} using techniques that seemingly don't translate well to the noncommutative setting.

\begin{theorem}
	Suppose $A$ and $B$ are commutative domains over an algebraically closed field $\k$ and $\k$ is algebraically closed in $A$ and $B$. If 
	$z\in A\otimes B$ is invertible, then $z = a\otimes b$ for some invertible elements $a\in A$ and $b\in B$.
\end{theorem}

The following two simplifications have been proven using Complex Analysis and Realization Theory:

\begin{proposition}
	Suppose $r\in \C\skf{\bbx}$ and $s\in \C\skf{\bby}$.
	If $(1\otimes 1 - r\otimes s)^{-1}\in \C\skf{\bbx}\otimes \C\skf{\bby}$ then either $r$ or $s$ is constant.
	
	Suppose $r_1,\dots, r_\tth\in \C\skf{\bbx}$, $C\skf{s_1\dots, s_\tth}$ is isomorphic as a skew field to $\C\skf{w_1,\dots, w_\tth}$.
	If $(1\otimes 1 - \sum_{i=1}^k r_i\otimes s_i)^{-1}\in \C\skf{\bbx}\otimes \C\skf{\bby}$ then $r_1,\dots, r_k$ are all constant.
\end{proposition}


The tensor product membership problem is strongly related to the rational automorphism problem as well, although that may require an 
understanding of domains.












%############################################################################
%############################################################################
%############################################################################


\section{Rational Automorphisms}
	\label{sec:RatAuts}
	
\begin{problem}
	Suppose $\bbr\in (\C\skf{x_1,\dots, x_\ttg})^\ttg$. Find an evaluation criterion that is necessary and sufficient for the induced map 
	$\rho:\C\skf{x_1,\dots, x_\ttg}\to\C\skf{x_1,\dots, x_\ttg}$ ($\rho(x_i) = \bbr_i$) to be an automorphism.
\end{problem}

An evaluation criterion is some condition on $\bbr$ when we treat it as a function.
For example, injective, surjective, etc.

\begin{conjecture}
	\label{conj:rats auts conj}
	Suppose $\bbr\in (\C\skf{x_1,\dots, x_\ttg})^\ttg$.
	The following are equivalent:
	\begin{enumerate}
		\item there exists a free, Euclidean open and Euclidean dense set $\Omega$ such that $\bbr\lvert_\Omega$ is injective;
		\item $J_\bbr$ is an invertible element of $M_\ttg(\C\skf{\bbx'}^{opp}\otimes \C\skf{\bbx})$;
		\item the induced map $\rho:\C\skf{\bbx}\to \C\skf{\bbx}$ given by $\rho(x_i) = \bbr_i$ is an automorphism.
	\end{enumerate}
\end{conjecture}


This is simply a best guess conjecture at the moment. Clearly, $(3)\Rightarrow (1),(2)$.
Condition $(1)$ is seemingly strange, but the na{\"i}ve attempt of requiring injective on its domain is insufficient.

\begin{example}
	Let $\bbr(x,y) = (x, y - x^2y)$.
	This induces a rational automorphism, however $\bbr(1,\alpha) = \bbr(1,\beta)$, hence $\bbr$ is not injective on its domain (in fact, the 
	points where it is not injective are exactly the points where $\bbr^{-1}$ is not defined).
	If $\Omega = \set{(X,Y)\in M_n(\C)^2 \, : \, \det(I_n - X^2)\neq 0}$, then $\Omega$ is free, Euclidean open and dense and $\bbr$ is 
	injective on $\Omega$.
\end{example}

Let us see a slightly harder example.

\begin{example}
	Let $\bbr(x,y) = (x, y - xyx)$.
	Naturally $\bbr$ is not injective on $\C^2$ since $\bbr(1,\alpha) = \bbr(1,\beta)$.
	Moreover, $\bbr$ does not induce a rational automorphism and this fact is a bit harder to see.
	
	The first observed reason why is found by using formal power series.
	Since the derivative of $\bbr$ is invertible on some free neighborhood of $0$, it must have a local inverse that we can find as a power 
	series.
	It turns out that
	\[
		f(x,y) = (x, \sum_{n=0}^\infty x^nyx^n)
	\]
	is the formal power series representation of $\bbr^{-1}$.
	If one tries to write down a realization for $f$, then a contradiction is eventually attained showing that $f$ is not a rational power 
	series.
	Thus, $\bbr$ does not induce a rational automorphism.
	This doesn't give us too much information at the moment, since it reveals little about the injectivity of $\bbr$.
	
	Our alternative approach is to use the idea behind the conception of hyporationals: the matrix identity $\vecc(AXB) = \vecc(X)(A^T\otimes 
	B)$.
	Let $f = (\bbr^{-1})_2$.
	Since $f(x,y)$ satisfies the equation $f(x,y) = y + xf(x,y)x$, we evaluate on a pair of matrices $(X,Y)$ (when it makes sense) and note 
	we have
	\[
		f(X,Y) = Y + Xf(X,Y)X.
	\]
	Taking the vectorization of both sides and rearranging, we have
	\[
		\vecc(f(X,Y))(I_n\otimes I_n - X^T\otimes X) = \vecc(I_n)(I_n\otimes Y).
	\]
	Multiplying on the right by an inverse we see
	\[
		\vecc(f(X,Y)) = \vecc(I_n)(I_n\otimes Y)\left(I_n\otimes I_n - X^T\otimes X\right)^{-1}.
	\]
	Thus, for any matrix $X$ where $(I_n\otimes I_n - X^T\otimes X)$ is invertible we have $(X,Y)$ is in the domain of $\bbr^{-1}$.
	As pointed out in \ref{sec:TPFSF}, the function $(1\otimes 1-x'\otimes x)^{-1}\notin \C\skf{x'}\otimes \C\skf{x}$.
	However, a consequence of its (first) proof is that there is no free Euclidean open and Euclidean dense set upon which $(1\otimes 
	1-x'\otimes x)$ is invertible.
	Thus, $\bbr$ fails to satisfy requirement $(1)$ from the Conjecture and $\bbr$ does not induce a rational automorphism.
	
	To see why $q = (1\otimes 1-x'\otimes x)$ fails to be injective on a ``big" set, suppose $\Omega$ is any nonempty free open set upon 
	which $q$ is invertible.
	If $\lambda$ and $\mu$ are any eigenvalues of $X$ then $\lambda\mu$ is an eigenvalue of $X^T\otimes X$ and $1-\lambda\mu$ is an 
	eigenvalue of $q(X^T,X)$.
	Hence, if $X\in \Omega$, then for each eigenvalue $\lambda$ of $X$, $\lambda^{-1}$ is not an eigenvalue of any matrix in $\Omega$.
	Thus, the eigenvalues of $q(X^T,X)$, taken over all $X\in \Omega$ partition the complex plane.
	
	For any $X\in M_n(\C)$ let $\sigma(X)$ denote its set of eigenvalues and for any $U\subset M_n(\C)$ let $\sigma(U) = \cup_{X\in U} 
	\sigma(X)$.
	Since $\Omega$ is assumed to open, $\Omega[n]$ is open and $\sigma(\Omega[n])$ must contain an open set.
	Hence, if $\lambda\in \sigma(\Omega[n])$ is nonzero, then there exists an open set $W$ containing $\lambda^{-1}$ such that $W\cap 
	\sigma(\Omega[n]) = \varnothing$.
	
	However, $\sigma^{-1}(W)$ contains an open set of matrices, thus $\Omega$ cannot be free, open and dense.	
\end{example}


Condition $(1)$ is currently a best guess (the density was used to invoke complex analytic methods).
Understanding the domains of elements of $\C\skf{\bbx'\lra \bbx}$ seems to be quite important.









%############################################################################
%############################################################################
%############################################################################


\section{Rational Derivatives - \textit{SOLVED!}}
	\label{sec:RatDeriv}
	
Let $\bbx = \set{x_1,\dots, x_\ttg}$ and $\bbh = \set{h_1,\dots, h_\ttg}$ be a sets of freely noncommuting indeterminates.
Let $\C\fpsx$ denote the formal power series in $\bbx$ and let $\C\skf{\bbx}_0$ denote the rational formal power series.
If $S\in \C\fpsx$ and $w$ is a word, then $[S,w]$ is the coefficient of $w$ appearing in $S$.
For any word $w$ and series $S\in \C\fpsx$, we let $w^{-1}S = \sum_{v\in \fax} [S,wv]v$.


\begin{proposition}
	Suppose $f\in \C\fpsx$.
	If $Df(\x)[\h] \in \C\skf{\bbx,\bbh}_0$ then $f\in \C\skf{\bbx}_0$.
\end{proposition}

\begin{proof}
	For each $1\leq i\leq \ttg$ we let $\partial_i f := Df(\x)[0,\dots,0,h_i,0,\dots,0]$ and note $\partial_i f\in \C\skf{\bbx,\bbh}_0$.
	Next, we write
	\[
		f = c_0 + \sum_{i=1}^\ttg x_i f_i
	\]
	where each $f_i\in \C\fpsx$.
	Hence,
	\[
		\partial_i f = h_if_i + \sum_{j=1}^\ttg x_j \partial_i f_j\in \C\skf{\bbx,\bbh}_0
	\]
	and it follows that $h_i^{-1}\partial_if_i = f_i\in \C\skf{\bbx,\bbh}_0$ since $\partial_i f_i$ is contained in a stable submodule.
	Therefore, $f = c_0 + \sum_{i=1}^\ttg x_i f_i\in \C\skf{\bbx}_0$ since each $f_i$ is rational.
\end{proof}


The argument below shows it for generalized series. Notation and ideas are pulled from \cite{Vol18}.
Notably, if $v$ is a word, $L_v S = \sum_{w\in \fax} [S,vw]$.

\begin{remark}
	Let $\cA = M_n(\C)$ for some $n$.
	If $f\in \cA\fpsx$ then
	\[
		f = f_0 + \sum_{i=1}^\ttg L_{x_i} f
	\]
	and
	\[
		Df(\x)[\h] = \sum_{i=1}^\ttg L_{h_i}Df(\x)[\h] + \sum_{i=1}^\ttg L_{x_i}Df(\x)[\h].
	\]
	If $f_i(\x,h_i) = L_{h_i}Df(\x)[\h]$, then $f_i(\x,x_i) = L_{x_i}f$.
	Hence, if $Df(\x)[\h]$ is rational, then so is $L_{h_i}Df(\x)[\h]$ and consequently so is $L_{x_i}f$.
	Therefore, $f$ is rational since it is a sum of rational series.	
\end{remark}












%############################################################################
%############################################################################
%############################################################################




\section{Artin approximation theorem}
	\label{sec:ArtinApprox}


\begin{problem}
	Suppose $\bbk\fpsx$ is the algebra of formal power series with freely noncommuting indeterminates $\bbx = \set{x_1,\dots, x_\ttg}$ over 
	the field $\bbk$.
	Suppose $\bby = \set{y_1,\dots, y_\tth}$ is a distinct set of freely noncommuting indeterminates and let 
	\[
		f(\x,\y) = 0
	\]
	be a system of polynomial equations in $\bbk\fralg{\bbx,\bby}$, and let $c$ be a positive integer.
	Then given a formal power series solution $\hat{\y}(\x)\in \bbk\fpsx$, there is an algebraic solution $\y(\x)$ consisting of algebraic 
	power series such that
	\[
		\hat{\y}(\x) \equiv \y(\x) \mod (\x)^c
	\]
\end{problem}


In the classical setting, this problem uses the Implicit function theorem and some refinements of the IFT.




\section{Miscellaneous Problems without exposition}
	\label{sec:AlgMisc}

Max pencil kernels and Makar-Limanov

NC Noether Problem.

Artin-Schreier










\begin{thebibliography}{6}

\bibitem[Coh06]{Cohn06}
P.M. Cohn.
\newblock {\em Free Ideal Ring and Localizations in General Rings}.
\newblock Cambridge University Press, 2006.


\bibitem[PW95]{PW95}
H.J. Park and C.~Woodburn.
\newblock An algorithmic proof of {S}uslin's stability theorem for polynomial
  rings.
\newblock {\em Journal of Algebra}, 178(1):277 -- 298, 1995.

\bibitem[Umi07]{U07}
U.~U. Umirbaev.
\newblock The {A}nick automorphism of free associative algebras.
\newblock {\em J. Reine Angew. Math.}, 605:165--178, 2007.


\bibitem[Sch85]{Sch85}
Schofield, A. H.
\newblock Representations of rings over skew fields.
\newblock {\em London Mathematical Society Lecture Note Series}, 1985.


\bibitem[Swe70]{Swe70}
Moss~Eisenberg Sweedler.
\newblock A units theorem applied to {H}opf algebras and {A}mitsur cohomology.
\newblock {\em American Journal of Mathematics}, 92(1):259--271, 1970.



\bibitem[Vol18]{Vol18}
Jurij Vol{\v c}i{\v c}.
\newblock Matrix coefficient realization theory of noncommutative rational functions.
\newblock {\em Journal of Algebra}, 499: 397 - 437, 2018.

\bibitem[Vol19]{Vol19}
Jurij Vol{\v c}i{\v c}.
\newblock Free Bertini's Theorem and applications.
\newblock \url{https://arxiv.org/pdf/1908.08948.pdf}.


\end{thebibliography}



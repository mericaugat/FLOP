%%%%%%%%%%%%%%%%%%%%%%%%%%%%%%%%%%%%%%%%%%%%
%	ENVIRONMENTS
%%%%%%%%%%%%%%%%%%%%%%%%%%%%%%%%%%%%%%%%%%%%
\newtheorem{theorem}{Theorem}[subsection]
\newtheorem{corollary}[theorem]{Corollary}
\newtheorem{lemma}[theorem]{Lemma}
\newtheorem{proposition}[theorem]{Proposition}
\newtheorem{prop}[theorem]{Proposition}
\newtheorem{question}{Question}
\newtheorem{claim}{Claim}%[theorem]
\theoremstyle{definition}
\newtheorem{definition}[theorem]{Definition}
\newtheorem{conjecture}[theorem]{Conjecture}
\newtheorem{guess}[theorem]{Guess}
%\newnumbered{conjecture}{Conjecture}%[theorem]
\newtheorem{remark}[theorem]{Remark}
%\newnumbered{example}[theorem]{Example}
%\theoremstyle{remark}
%\newtheorem{rem}[theorem]{Remark}
%\theoremstyle{remark}
%\newtheorem{remark}[theorem]{Remark}
\theoremstyle{definition}
\newtheorem{example}[theorem]{Example}
%\theoremstyle{definition}
%\newnumbered{examplex}{Example}[theorem]
%\newenvironment{example}
%  {\pushQED{\qed}\renewcommand{\qedsymbol}{$\Diamond$}\examplex}
%  {\popQED\endexamplex}
%%\theoremstyle{definition}
%\newnumbered{remarkx}{Remark}[theorem]
%\newenvironment{remark}
%{\pushQED{\qed}\renewcommand{\qedsymbol}{$\Diamond$}\remarkx}
%{\popQED\endremarkx}

%\newtheorem*{rem*}{Remark}
%\newtheorem*{theorem*}{Theorem}
%\newtheorem*{claim*}{Claim}
%\newtheorem*{proposition*}{Proposition}
%\newtheorem*{definition*}{Definition}
%\newtheorem*{lemma*}{Lemma}
\newtheorem{fact*}{Fact}
%\newtheorem*{question*}{Question}
\newtheorem{note}[theorem]{Note}

\newtheorem{thmx}{Theorem}
\renewcommand{\thethmx}{\Alph{thmx}} % "letter-numbered" theorems

\newcounter{tmp}
\newcounter{letter}


% MATH -------------------------------------------------------------------
\DeclareMathOperator{\RE}{Re}
\DeclareMathOperator{\IM}{Im}
\DeclareMathOperator{\intr}{int}
\DeclareMathOperator{\dist}{dist}
\DeclareMathOperator{\dom}{dom}
\DeclareMathOperator{\diag}{diag}
\DeclareMathOperator\re{\mathrm {Re~}}
\DeclareMathOperator\im{\mathrm {Im~}}
\DeclareMathOperator{\rg}{range}
\newcommand{\NP}[1]{\mathrm{NP}(#1)}
\newcommand{\ANP}[1]{\mathrm{ANP}(#1)}
\newcommand{\finv}[1]{{#1}^{\bm{\dagger}/2}}
\newcommand{\alg}{\mathrm{alg}}
\DeclareMathOperator\vecc{{\bf vec}}
\DeclareMathOperator\verr{{\bf ver}}
\DeclareMathOperator\id{{id}}
%\newcommand\half{\tfrac{1}{2}}
\newcommand\half{1\hspace{-0.1em}/\hspace{-0.08em}2}
\newcommand{\eps}{\varepsilon}


%%%%%%%%%%%%%%%%%%%%%%%%%%%%%%%%%%%%%%%%%%%%
%	MATHCAL
%%%%%%%%%%%%%%%%%%%%%%%%%%%%%%%%%%%%%%%%%%%%
\newcommand{\A}{\mathcal{A}}
\newcommand{\B}{\mathcal{B}}
\newcommand{\vecspace}{\mathcal{V}}
\newcommand{\J}{\mathcal{J}}
\newcommand{\M}{\mathcal{M}}
\newcommand{\Z}{\mathbb{Z}}
\newcommand{\N}{\mathbb{N}}
\newcommand{\T}{\mathbb{T}}
\newcommand{\W}{\mathcal{W}}
\newcommand{\X}{\mathcal{X}}
\newcommand{\cC}{\mathcal{C}}
\newcommand{\cH}{\mathcal{H}}
\newcommand{\cK}{\mathcal{K}}
\newcommand{\cA}{\mathcal{A}}
\newcommand{\cE}{\mathcal{E}}
\newcommand{\cG}{\mathcal{G}}
\newcommand{\cD}{\mathcal{D}}
\newcommand{\cP}{\mathcal{P}}
\newcommand{\cQ}{\mathcal{Q}}
\newcommand{\cR}{\mathcal{R}}
\newcommand{\cV}{\mathcal{V}}
\newcommand{\Polar}{\mathcal{P}_{\s}}
\newcommand{\Poly}{\mathcal{P}(E)}
\newcommand{\Lop}{\mathcal{L}}
\newcommand\RR{\mathcal{R}}
\newcommand\BH{B(\mathcal{H})}

%%%%%%%%%%%%%%%%%%%%%%%%%%%%%%%%%%%%%%%%%%%%
%	MATHBB
%%%%%%%%%%%%%%%%%%%%%%%%%%%%%%%%%%%%%%%%%%%%
\newcommand{\Complex}{\mathbb{C}}
\newcommand{\C}{\mathbb{C}}
\newcommand{\D}{\mathbb{D}}
\newcommand{\Field}{\mathbb{F}}
\newcommand{\bbG}{\mathbbm{G}}
\newcommand{\K}{\mathbb{K}}
\newcommand{\R}{\mathbb{R}}
\newcommand{\RPlus}{\Real^{+}}



\newcommand{\bba}{\mathbbm{a}}
\newcommand{\bbd}{\mathbbm{d}}
\renewcommand{\k}{\mathbbm{k}}
\newcommand{\bbn}{\mathbbm{n}}
\newcommand{\bbp}{\mathbbm{p}}
\newcommand{\bbq}{\reflectbox{$\mathbbm{p}$}}
\newcommand{\bbr}{\mathbbm{r}}
\newcommand{\bbs}{\mathbbm{s}}
\newcommand{\bbu}{\mathbbm{u}}
\newcommand{\bbv}{\mathbbm{v}}
\newcommand{\bbw}{\mathbbm{w}}
\newcommand{\bbx}{\mathbbm{x}}
\newcommand{\bby}{\mathbbm{y}}
\newcommand{\bbz}{\mathbbm{z}}



\newcommand{\ew}{\mathbbm{1}}







%%%%%%%%%%%%%%%%%%%%%%%%%%%%%%%%%%%%%%%%%%%%
%	FRAKTURE
%%%%%%%%%%%%%%%%%%%%%%%%%%%%%%%%%%%%%%%%%%%%
\newcommand\mfq{\mathfrak{q}}
\newcommand\mfG{\mathfrak{G}}
\newcommand\Choi{\mathfrak{C}}
\newcommand\mfa{\mathfrak{a}}




%%%%%%%%%%%%%%%%%%%%%%%%%%%%%%%%%%%%%%%%%%%%
%	BOLD
%%%%%%%%%%%%%%%%%%%%%%%%%%%%%%%%%%%%%%%%%%%%
\newcommand{\x}{\bm{x}}
\newcommand{\y}{\bm{y}}
\newcommand{\z}{\bm{z}}
\newcommand{\h}{\bm{h}}
\newcommand{\w}{\bm{w}}
\renewcommand{\u}{\bm{u}}
\newcommand{\s}{\bm{s}}


%%%%%%%%%%%%%%%%%%%%%%%%%%%%%%%%%%%%%%%%%%%%
%	UPRIGHT TEXT
%%%%%%%%%%%%%%%%%%%%%%%%%%%%%%%%%%%%%%%%%%%%
\newcommand{\GM}{\textup{GM}}
\newcommand{\UD}{\textup{UD}}
\newcommand{\Aut}[1]{\textup{Aut}(#1)}



%%%%%%%%%%%%%%%%%%%%%%%%%%%%%%%%%%%%%%%%%%%%
%	GREEK
%%%%%%%%%%%%%%%%%%%%%%%%%%%%%%%%%%%%%%%%%%%%
\newcommand{\vp}{\varphi}
\newcommand{\ph}{\varphi}
\newcommand\al{\alpha}
\newcommand\ga{\gamma}
\newcommand\de{\delta}
\newcommand\ep{\varepsilon}
\newcommand\La{\Lambda}
\newcommand\la{\lambda}
\newcommand\up{\upsilon}
\newcommand\si{\sigma}




\newcommand{\cc}[1]{\overline{#1}}
\newcommand{\abs}[1]{{\left\vert#1\right\vert}}
\newcommand{\set}[1]{{\left\{#1\right\}}}
\newcommand{\seq}[1]{{\left<#1\right>}}
\newcommand{\norm}[1]{{\left\Vert#1\right\Vert}}
\newcommand{\tr}{\operatorname{tr}}
\newcommand{\trp}{\operatorname{Trp}}
\newcommand{\ran}[1]{\operatorname{ran}#1}
\newcommand{\spn}[1]{\operatorname{span}\left\{ #1 \right\}}
\newcommand{\nt}{\stackrel{\mathrm {nt}}{\to}}
\newcommand{\pnt}{\xrightarrow{pnt}}
\newcommand{\vr}[1]{\bold{#1}}
\newcommand{\uvr}[1]{\hat{\bold{#1}}}
\newcommand{\ip}[2]{\left\langle #1, #2 \right\rangle}
\newcommand{\wrd}[1]{\left\langle #1\right\rangle}
\newcommand{\ad}{^\ast}
\newcommand{\inv}{^{-1}}
\newcommand{\pinv}{^{\dagger}}
\newcommand{\adinv}{^{\ast -1}}
\newcommand{\invad}{^{-1 \ast}}

\newcommand{\til}{\raise.17ex\hbox{$\scriptstyle\mathtt{\sim}$}}
\newcommand{\res}[1]{\big\vert_{#1}}





\newcommand\Htau{\mathbb{H}(\tau)}






\newcommand\ds{\displaystyle}

\newcommand{\df}[1]{{\it{#1}}{\index{#1}}}
\newcommand\ii{\mathrm i}







\newcommand\blue{\color{blue}}
\newcommand\black{\color{black}}
\newcommand\red{\color{red}}

\newcommand\nn{\nonumber}



%%%%%%%%%%%%%%%%%%%%%%%%%%%%%%%%%%%%%%%%%%%%
%	MATRICES, ARRAYS, PIECEWISE AND ALIGNS
%%%%%%%%%%%%%%%%%%%%%%%%%%%%%%%%%%%%%%%%%%%%

\newcommand{\bbm}{\left[ \begin{smallmatrix}}
\newcommand{\ebm}{\end{smallmatrix} \right]}

\newcommand{\bpm}{\begin{pmatrix}}
\newcommand{\epm}{\end{pmatrix}}
\newcommand{\bal}{\begin{align*}}
\newcommand{\eal}{\end{align*}}

\newcommand{\kp}[2]
{
	\begin{matrix}
		#1 \\
		#2
	\end{matrix}
}


\newcommand{\vectwo}[2]
{
   \begin{pmatrix} #1 \\ #2 \end{pmatrix}
}
\newcommand{\vecthree}[3]
{
   \begin{pmatrix} #1 \\ #2 \\ #3 \end{pmatrix}
}

\newcommand{\twopartdef}[4]
{
	\left\{
		\begin{array}{ll}
			#1 & \mbox{if } #2 \\
			#3 & \mbox{if } #4
		\end{array}
	\right.
}
\newcommand{\threepartdef}[6]
{
	\left\{
		\begin{array}{lll}
			#1 & \mbox{if } #2 \\
			#3 & \mbox{if } #4 \\
			#5 & \mbox{if } #6
		\end{array}
	\right.
}
\newcommand{\tensor}[2]{\text{ }{\begin{smallmatrix} #1 \\ \otimes\\ #2\end{smallmatrix}}\text{  }}
\newcommand{\flattensor}[2]{#1 \otimes #2}
\newcommand{\bigtensor}[2]{\text{ }\begin{matrix} #1 \\{\otimes}\\ #2\end{matrix}\text{  }}
\newcommand\smallmath[2]{#1{\raisebox{\dimexpr \fontdimen 22 \textfont 2
      - \fontdimen 22 \scriptfont 2 \relax}{$\scriptstyle #2$}}}
\newcommand{\sot}{\!\smallmath\mathbin\otimes\!}


\numberwithin{equation}{section}
\newcommand\noin{\noindent}


\newlength{\Mheight}
\newlength{\cwidth}
\newcommand{\mc}{\settoheight{\Mheight}{M}\settowidth{\cwidth}{c}M\parbox[b][\Mheight][t]{\cwidth}{c}}
\newcommand{\foot}[1]{\footnote{#1}}
\newcommand{\dfn}[1]{{\bf #1}\index{#1}}
\newcommand{\tir}[1]{\tensor{#1}{I}}
\newcommand{\tidr}[1]{\tensor{#1}{\mathrm{id}}}

\newcommand\psub{\mathrel{%
  \ooalign{\raise0.2ex\hbox{$\subset$}\cr\hidewidth\raise-0.8ex\hbox{\box{0.9}{$\sim$}}\hidewidth\cr}}}
\newcommand\piso{\mathrel{%
  \ooalign{\raise0.2ex\hbox{$=$}\cr\hidewidth\raise-0.5ex\hbox{\scalebox{0.9}{$\sim$}}\hidewidth\cr}}}




\newcommand*{\defeq}{\mathrel{\vcenter{\baselineskip0.6ex \lineskiplimit0pt
                     \hbox{\small.}\hbox{\small.}}}%
                     =}


\makeatletter
\def\moverlay{\mathpalette\mov@rlay}
\def\mov@rlay#1#2{\leavevmode\vtop{
                \baselineskip\z@skip \lineskiplimit-\maxdimen
                \ialign{\hfil$#1##$\hfil\cr#2\crcr}}}
\makeatother





%%%%%%%%%%%%%%%%%%%%%%%%%%%%%%%%%%%%%%%%%%%%%%%%%%%%%%%%%%%%%%%%%%%
%		\llangle and \rrangle
\makeatletter
\DeclareFontFamily{OMX}{MnSymbolE}{}
\DeclareSymbolFont{MnLargeSymbols}{OMX}{MnSymbolE}{m}{n}
\SetSymbolFont{MnLargeSymbols}{bold}{OMX}{MnSymbolE}{b}{n}
\DeclareFontShape{OMX}{MnSymbolE}{m}{n}{
    <-6>  MnSymbolE5
   <6-7>  MnSymbolE6
   <7-8>  MnSymbolE7
   <8-9>  MnSymbolE8
   <9-10> MnSymbolE9
  <10-12> MnSymbolE10
  <12->   MnSymbolE12
}{}
\DeclareFontShape{OMX}{MnSymbolE}{b}{n}{
    <-6>  MnSymbolE-Bold5
   <6-7>  MnSymbolE-Bold6
   <7-8>  MnSymbolE-Bold7
   <8-9>  MnSymbolE-Bold8
   <9-10> MnSymbolE-Bold9
  <10-12> MnSymbolE-Bold10
  <12->   MnSymbolE-Bold12
}{}

\let\llangle\@undefined
\let\rrangle\@undefined
\DeclareMathDelimiter{\llangle}{\mathopen}%
                     {MnLargeSymbols}{'164}{MnLargeSymbols}{'164}
\DeclareMathDelimiter{\rrangle}{\mathclose}%
                     {MnLargeSymbols}{'171}{MnLargeSymbols}{'171}
\makeatother

%		\llangle and \rrangle
%%%%%%%%%%%%%%%%%%%%%%%%%%%%%%%%%%%%%%%%%%%%%%%%%%%%%%%%%%%%%%%%%%%%


\newcommand*{\mpsymbol}{
        \!\protect\raisebox{0.534ex}{\protect\scalebox{.28}{
                        \begin{tikzpicture}[line width=0.6ex]
                        \draw[arrows={<->}]
                        (0,0) -- (5ex,0);
                        \end{tikzpicture}
                }}\!
        }

\newcommand{\lra}{\mpsymbol}
\renewcommand{\mp}{\textrm{mp}}

\newcommand{\fps}[1]{\llangle #1 \rrangle}
\newcommand{\fpsx}{\fps{\bbx}}
\newcommand{\fralg}[1]{\langle #1 \rangle}
\newcommand{\fax}{\fralg{\bbx}}


\newcommand{\plangle}{\moverlay{(\cr<}}
\newcommand{\prangle}{\moverlay{)\cr>}}

\newcommand{\flangle}{\plangle\hphantom{_o}\mathllap{\plangle}}
\newcommand{\frangle}{\mathrlap{\prangle}\hphantom{_o}\prangle}

\newcommand{\skf}[1]{\plangle #1 \prangle}
\newcommand{\fls}[1]{\flangle #1 \frangle}


%\newcommand{\hypj}[1]{J^{\text{hyp}}_{#1}}
\newcommand{\hypj}[1]{J^{\mathbbm{H}}_{#1}}
%\newcommand{\hypj}[1]{\moverlay{\triangle \cr \raisebox{0.025ex}{\scaleobj{0.775}{\times}}   }_{#1}}
%\newcommand{\hypmat}{\moverlay{\triangle \cr \raisebox{0.025ex}{\scaleobj{0.775}{\times}}   }}

%\makeatletter
%\def\moverlay{\mathpalette\mov@rlay}
%\def\mov@rlay#1#2{\leavevmode\vtop{
%                \baselineskip\z@skip \lineskiplimit-\maxdimen
%                \ialign{\hfil$#1##$\hfil\cr#2\crcr}}}
%\makeatother





\newcommand{\meric}{\red \sc}




%\begin Shrug
\newcommand{\shrug}[1][]{%
\begin{tikzpicture}[baseline,x=0.8\ht\strutbox,y=0.8\ht\strutbox,line width=0.125ex,#1]
\def\arm{(-2.5,0.95) to (-2,0.95) (-1.9,1) to (-1.5,0) (-1.35,0) to (-0.8,0)};
\draw \arm;
\draw[xscale=-1] \arm;
\def\headpart{(0.6,0) arc[start angle=-40, end angle=40,x radius=0.6,y radius=0.8]};
\draw \headpart;
\draw[xscale=-1] \headpart;
\def\eye{(-0.075,0.15) .. controls (0.02,0) .. (0.075,-0.15)};
\draw[shift={(-0.3,0.8)}] \eye;
\draw[shift={(0,0.85)}] \eye;
% draw mouth
\draw (-0.1,0.2) to [out=15,in=-100] (0.4,0.95); 
\end{tikzpicture}}
%\end shrug




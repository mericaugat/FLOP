\documentclass[oneside]{amsbook}



\usepackage{graphicx}

%\allowdisplaybreaks[1]
\pdfoutput=1

\usepackage{amssymb,mathrsfs, amsmath,amsfonts,verbatim}%, amsthm}
\usepackage{mathtools}
\usepackage{bm}
\usepackage[usenames]{color}

\usepackage{scalerel}

\usepackage{xfrac}

\usepackage{xcolor}

\definecolor{orange}{rgb}{1,0.5,0}



%\usepackage{showkeys}
%\usepackage{showtags}

%\usepackage{enumerate}

\usepackage{tikz}


\usepackage{microtype}
\usepackage{stackengine}
%\usepackage{upgreek}
\usepackage{cmap}
\usepackage{bbm}
%\usepackage{thmtools}
%\usepackage{thm-restate}


\usepackage{etoolbox}
\patchcmd{\thebibliography}{\chapter*}{\section*}{}{}


\usepackage[colorlinks=true,allcolors=blue]{hyperref}

\include{FLOP-Commands}

\showboxdepth=\maxdimen
\showboxbreadth=\maxdimen




\numberwithin{equation}{section}
\setcounter{secnumdepth}{3}



\makeindex

\allowdisplaybreaks

\title{FLOP -- Free List of Open Problems -- \today}




\begin{document}

\begin{center}
	\textbf{\huge{FLOP -- Free List of Open Problems}}
	
%	\huge{\today}
\end{center}
\tableofcontents





%############################################################################
%############################################################################
%############################################################################

%					ALGEBRAIC CHAPTER

%############################################################################
%############################################################################
%############################################################################


\chapter{Algebraic Problems}





\section{Algebra Notation}

\begin{center}
\begin{tabular}{l c l}
	$\C\fralg{x_1,\dots,x_\ttg}$ &  & free $\C$-algebra generated by $x_1,\dots, x_{\ttg}$ \\
	$\C\skf{x_1,\dots,x_\ttg}$ &  & free skew field (nc rationals) of $\C\fralg{x_1,\dots,x_\ttg}$  \\
	$\C\skf{\bbx}_0$ &  & algebra of rationals regular at $0$ \\
	$\C\skf{\bbx \lra \bby}$ &  & universal skew field of fractions of $\C\skf{\bbx}\otimes \C\skf{\bby}$
\end{tabular}
\end{center}










%############################################################################
%############################################################################
%############################################################################

\section{Factoring Invertible Matrices over $\C\fralg{x}\otimes \C\fralg{y}$}
	\label{sec:Elem_Mats}

\begin{problem}
	Suppose $A\in M_n(\C\fralg{\bbx}\otimes \C\fralg{\bby})$. If $A$ is invertible over the $\C$-algebra $\C\fralg{\bbx}\otimes 
	\C\fralg{\bby}$ then do there exist $D\in M_n(\C)$ and $E_1,\dots, E_k\in M_n(\C\fralg{\bbx}\otimes \C\fralg{\bby})$ such that $D$ is 
	diagonal, each $E_i$ is an elementary matrix and $A = DE_1\dots E_k$?
\end{problem}

\begin{remark}
	This problem is false when $n=2$. Cohn gave the following matrix that is not a product of elementary matrices over $\C[x]\otimes \C[y]$:
	\[
		\bpm 1\otimes 1 + x\otimes y & 1\otimes y^2 \\ -x^2\otimes 1 & 1\otimes 1 - x\otimes y \epm.
	\]
\end{remark}

The above counterexample is nonexistent when our matrices are over $\C\fralg{\bbx}$ instead of $\C\fralg{\bbx}\otimes \C\fralg{\bby}$. The 
proof of this uses results from \cite{Cohn06}.

\begin{theorem}
	If $A\in M_n(\C\fralg{\bbx})$ is invertible, then there exist $D\in M_n(\C)$ and $E_1,\dots, E_k\in M_n(\C\fralg{\bbx})$ such that $D$ is 
	diagonal, each $E_i$ is an elementary matrix and $A = DE_1\dots E_k$.
\end{theorem}

The Cohn counterexample is essentially the only issue:

\begin{theorem}[Suslin's Stability Theorem]
	Suppose $A\in M_n(\C[t_1,\dots, t_\ttg])$ where $n\geq 3$. If $\det(A) = 1$ then there exist $E_1,\dots, E_k\in M_n(\C[t_1,\dots, 
	t_\ttg])$ such that $A = E_1\dots E_k$.
\end{theorem}

An ``algorithmic" proof of the above theorem can be found in \cite{PW95}.


\begin{remark}
	An idea related to this is the notion of Jacobian Tame, see \ref{sec:Jac Tame}.
\end{remark}










%############################################################################
%############################################################################
%############################################################################




\section{Free L{\"u}roth Theorem}
	\label{sec:Luroth}
	
\begin{problem}[Free L{\"u}roth Theorem]
Suppose $\k$ is an algebraically closed infinite field and $x_1,\dots, x_g$ are freely noncommuting indeterminates ($g\geq 2$).
If $\k\subsetneq D\subsetneq \k\skf{x_1,\dots, x_\ttg}$ is a subfield, then do there exist $q_1,\dots, q_\tth\in \C\skf{x_1,\dots, x_\ttg}$ 
such that $D = \C\skf{q_1, \dots, q_\tth}$?

In other words, is every non-trivial subfield of $\C\skf{\bbx}$ a free skew field?
\end{problem}


\begin{theorem}[L{\"u}roth's Theorem]

Suppose $\k$ is a field. If $\k \subsetneq D \subsetneq \k(t)$ is a subfield, then there exists $q\in \k(t)$ such that $D = \k(q(t))$.

\end{theorem}

\begin{remark}
	If $\k$ is algebraically closed and infinite, then L{\"u}roth's Theorem holds as well for $\k(t_1,t_2)$.
	
	On the other hand, there are counterexamples to L{\"u}roth's Theorem for $\k(t_1,t_2,t_3)$.
\end{remark}

\begin{remark}
	Schofield \cite{Sch85} shows that if $f,g\in \C\skf{\bbx}$, then either $[f,g] = 0$ or $\C\skf{f,g}$ is free.
\end{remark}














%############################################################################
%############################################################################
%############################################################################



\section{Free Bertini Theorem - \textit{SOLVED!}}
	\label{sec:FreeBertini}

\noindent Great result solved in \cite{Vol19}


We say $f\in \bbk\fralg{\bbx}$ \textbf{factors} if $f = f_1f_2$ for some nonconstant $f_1,f_2\in \bbk\fralg{\bbx}$. Otherwise $f$ is 
\textbf{irreducible} over $\bbk$.
A nonconstant $f\in \bbk\fralg{\bbx}$ is \textbf{composite} if there exist $h\in \bbk\fralg{\bbx}$ and a univariate polynomial $p\in \bbk[t]$ 
such that $\deg(p)>1$ and $f = p\circ h = p(h)$.


\begin{theorem}
	If $f\in \bbk\fralg{\bbx}\setminus \bbk$ is not composite, then $f - \lambda$ is irreducible over $\overline{\bbk}$ for all but finitely 
	many $\lambda\in\overline{\bbk}$.
\end{theorem}














%############################################################################
%############################################################################
%############################################################################

\section{Rational Bertini Theorem}
	\label{sec:RatBertini}
	
	
	
A nonconstant $f\in \bbk\skf{\bbx}$ is \textbf{composite} if there exist $h\in \bbk\skf{\bbx}$ and a univariate rational $p\in \bbk(t)$ such 
that $p$ is not M{\"o}bius and $f = p\circ h = p(h)$.

\begin{problem}
	A rational function $r$ is not composite if and only if the hypersurfaces
	\[
		\set{X\in \dom_n(r) \, : \, \det(r(X) - \lambda I) = 0} \subset M_n(\C)^\ttg
	\]
	are irreducible for all but finitely many $n\in \N$ and $\lambda\in \C$.
\end{problem}











%############################################################################
%############################################################################
%############################################################################

\section{Jacobian Tame}
	\label{sec:Jac Tame}

An automorphism $\tau$ of the free algebra $\C\fralg{x_1,\dots, x_\ttg}$ is {\bf elementary} if $\tau:x_i \mapsto x_i$ 
	for $i\neq j$ and $\tau:x_j \mapsto cx_j + f$, where $c\in \C$ and $f\in \C\fralg{x_1,\dots, x_{j-1}, x_{j+1},\dots, x_\ttg}$.
We say an automorphism is {\bf tame} if it is a composition of elementary automorphisms.
If the automorphism is not tame, then it is {\bf wild}.

In \cite{U07} it is shown that the Anick automorphism, 
\[
	\delta(x,y,z) = (x + z(xz - zy), y + (xz - zy)z, z)\in \C\fralg{x,y,z}
\]
is wild.
Looking at it, the Anick automorphism doesn't seem particularly ``wild," so maybe this can be improved?
Perhaps it is something about Jacobian matrices?

\begin{definition}
	We say an automorphism $\tau$ of $\C\fralg{x_1,\dots, x_\ttg}$ is \textbf{Jacobian Tame} if
	$J_\tau = DE_1\dots E_k$, where $D\in M_\ttg(\C\fralg{\bbx'}^{opp}\otimes \C\fralg{\bbx})$ is diagonal and
	$E_1,\dots, E_k\in M_\ttg(\C\fralg{\bbx'}^{opp}\otimes \C\fralg{\bbx})$ are elementary matrices.
	
	If such a factorization does not exist, then we say $\tau$ is \textbf{Jacobian wild}.
\end{definition}


\begin{problem}
	Are there any Jacobian wild automorphisms of the free algebra?
	The only known (to me) example of a wild automorphism of $\C\fralg{x,y,z}$ is Jacobian tame (this is explained below).
	
	In the commutative case there are no Jacobian wild automorphisms.
	Every automorphism of $\C[t_1,t_2]$ is tame, hence Jacobian tame.
	If $\phi$ is an automorphism of $\C[t_1,\dots, t_\ttg]$ with $\ttg>2$, then its Jacobian matrix
	$J_\phi\in M_\ttg(\C[t_1,\dots, t_\ttg,t_1',\dots,t_\ttg'])$ factors into such a product by Suslin's Stability Theorem (see 
	\ref{sec:Elem_Mats}), thus is Jacobian tame.
\end{problem}

I will be using the transposed Jacobian matrix: $J_\tau \in M_\ttg(\C\fralg{\bbx'}^{opp}\otimes \C\fralg{\bbx})$ where the $i^\text{th}$ 
column of $J_\tau$ corresponds to the derivatives of $\tau_i$.

It turns out that the Anick automorphism 
\[
	\delta(x,y,z) = (x + z(xz - zy), y + (xz - zy)z, z)\in \C\fralg{x,y,z}
\]
has a Jacobian matrix that can be written as a product of elementary matrices.
The Jacobian matrix of $\delta$ is
\[
	J_\delta = 
	\bpm
		1\otimes 1 + z'\otimes z & 1\otimes z^2 & 0 \\
		-(z')^2\otimes 1 & 1 - z'\otimes z & 0 \\
		\zeta_1 & \zeta_2 & 1
	\epm
\]
where $\zeta_1 = 1\otimes xz+ x'z'\otimes 1 - 1\otimes zy - z'\otimes y$ and 
	$\zeta_2 = x\otimes z + z'x'\otimes 1 - 1\otimes yz - y'z'\otimes 1$.
Let $E_{i,j}(\alpha) = I + \alpha e_{i,j}$ ($e_{i,j}$ has a $1$ in the $i,j$ entry and 0's elsewhere) and observe
\[
	J_\delta E_{3,1}(-\zeta_1)E_{3,2}(-\zeta_2) = 
	\bpm
		1\otimes 1 + z'\otimes z & 1\otimes z^2 & 0 \\
		-(z')^2\otimes 1 & 1 - z'\otimes z & 0 \\
		0 & 0 & 1
	\epm.
\]
In 1966, Cohn proved that
\[
	\mathfrak{C} = 
	\bpm
		1\otimes 1 + z'\otimes z & 1\otimes z^2\\
		-(z')^2\otimes 1 & 1 - z'\otimes z
	\epm
\]
cannot be written as a product of elementary matrices over $\mathbb{F}[z'\otimes 1,1\otimes z]$.
This is a crucial aspect of Umirbaev's (\cite{U07}) proof that the Anick automorphism is wild.
However, Park and Woodburn \cite{PW95} give a decomposition of $(\begin{smallmatrix} \mathfrak{C} & 0 \\ 0 & 1 \end{smallmatrix})$ 
	into a product of elementary matrices:
\begin{align*}
	J_\delta &E_{3,1}(-\zeta_1)E_{3,2}(-\zeta_2) =
	\bpm
		1\otimes 1 + z'\otimes z & 1\otimes z^2 & 0 \\
		-(z')^2\otimes 1 & 1 - z'\otimes z & 0 \\
		0 & 0 & 1
	\epm\\
	= &	
	E_{2,3}(-z'\otimes 1)E_{1,3}(1\otimes z)E_{3,2}(1\otimes z)E_{3,1}(z'\otimes 1)\\
	& E_{2,3}(z'\otimes 1)E_{1,3}(-1\otimes z)E_{3,2}(-1\otimes z)E_{3,1}(-z'\otimes 1).
\end{align*}
Thus
\begin{align*}
	J_\delta = & E_{2,3}(-z'\otimes 1)E_{1,3}(1\otimes z)E_{3,2}(1\otimes z)E_{3,1}(z'\otimes 1)\\
	&E_{2,3}(z'\otimes 1)E_{1,3}(-1\otimes z)E_{3,2}(-1\otimes z)E_{3,1}(-z'\otimes 1)E_{3,2}(\zeta_2)E_{3,1}(\zeta_1).
\end{align*}
Thus, the Jacobian matrix of $\delta$ factors as a product of elementary matrices even though it is wild.


\vspace{1em}



Naturally, a positive answer to the Problem in \ref{sec:Elem_Mats} would show that every automorphism is Jacobian Tame.
On the other hand, a negative answer to the Problem in \ref{sec:Elem_Mats} would only serve to complicate things since a Jacobian matrix will 
certainly look quite different from a generic matrix in $\GL_\ttg(\C\fralg{\bbx'}^{opp}\otimes \C\fralg{\bbx})$.















%############################################################################
%############################################################################
%############################################################################

\section{Tensor Product of Free Skew Fields}
	\label{sec:TPFSF}
	

\begin{problem}
	Classify the invertible elements of $\C\skf{\bbx}\otimes \C\skf{\bby}$.
	The expectation is that if $Q$ and $Q^{-1}$ are in $\C\skf{\bbx}\otimes \C\skf{\bby}$, then $Q = r\otimes s$, for some nonzero elements 
	$r\in \C\skf{\bbx}$ and $s\in \C\skf{\bby}$.
\end{problem}

The following was proven in \cite{Swe70} using techniques that seemingly don't translate well to the noncommutative setting.

\begin{theorem}
	Suppose $A$ and $B$ are commutative domains over an algebraically closed field $\k$ and $\k$ is algebraically closed in $A$ and $B$. If 
	$z\in A\otimes B$ is invertible, then $z = a\otimes b$ for some invertible elements $a\in A$ and $b\in B$.
\end{theorem}

The following two simplifications have been proven using Complex Analysis and Realization Theory:

\begin{proposition}
	Suppose $r\in \C\skf{\bbx}$ and $s\in \C\skf{\bby}$.
	If $(1\otimes 1 - r\otimes s)^{-1}\in \C\skf{\bbx}\otimes \C\skf{\bby}$ then either $r$ or $s$ is constant.
	
	Suppose $r_1,\dots, r_\tth\in \C\skf{\bbx}$, $C\skf{s_1\dots, s_\tth}$ is isomorphic as a skew field to $\C\skf{w_1,\dots, w_\tth}$.
	If $(1\otimes 1 - \sum_{i=1}^k r_i\otimes s_i)^{-1}\in \C\skf{\bbx}\otimes \C\skf{\bby}$ then $r_1,\dots, r_k$ are all constant.
\end{proposition}


The tensor product membership problem is strongly related to the rational automorphism problem as well, although that may require an 
understanding of domains.












%############################################################################
%############################################################################
%############################################################################


\section{Rational Automorphisms}
	\label{sec:RatAuts}
	
\begin{problem}
	Suppose $\bbr\in (\C\skf{x_1,\dots, x_\ttg})^\ttg$. Find an evaluation criterion that is necessary and sufficient for the induced map 
	$\rho:\C\skf{x_1,\dots, x_\ttg}\to\C\skf{x_1,\dots, x_\ttg}$ ($\rho(x_i) = \bbr_i$) to be an automorphism.
\end{problem}

An evaluation criterion is some condition on $\bbr$ when we treat it as a function.
For example, injective, surjective, etc.

\begin{conjecture}
	\label{conj:rats auts conj}
	Suppose $\bbr\in (\C\skf{x_1,\dots, x_\ttg})^\ttg$.
	The following are equivalent:
	\begin{enumerate}
		\item there exists a free, Euclidean open and Euclidean dense set $\Omega$ such that $\bbr\lvert_\Omega$ is injective;
		\item $J_\bbr$ is an invertible element of $M_\ttg(\C\skf{\bbx'}^{opp}\otimes \C\skf{\bbx})$;
		\item the induced map $\rho:\C\skf{\bbx}\to \C\skf{\bbx}$ given by $\rho(x_i) = \bbr_i$ is an automorphism.
	\end{enumerate}
\end{conjecture}


This is simply a best guess conjecture at the moment. Clearly, $(3)\Rightarrow (1),(2)$.
Condition $(1)$ is seemingly strange, but the na{\"i}ve attempt of requiring injective on its domain is insufficient.

\begin{example}
	Let $\bbr(x,y) = (x, y - x^2y)$.
	This induces a rational automorphism, however $\bbr(1,\alpha) = \bbr(1,\beta)$, hence $\bbr$ is not injective on its domain (in fact, the 
	points where it is not injective are exactly the points where $\bbr^{-1}$ is not defined).
	If $\Omega = \set{(X,Y)\in M_n(\C)^2 \, : \, \det(I_n - X^2)\neq 0}$, then $\Omega$ is free, Euclidean open and dense and $\bbr$ is 
	injective on $\Omega$.
\end{example}

Let us see a slightly harder example.

\begin{example}
	Let $\bbr(x,y) = (x, y - xyx)$.
	Naturally $\bbr$ is not injective on $\C^2$ since $\bbr(1,\alpha) = \bbr(1,\beta)$.
	Moreover, $\bbr$ does not induce a rational automorphism and this fact is a bit harder to see.
	
	The first observed reason why is found by using formal power series.
	Since the derivative of $\bbr$ is invertible on some free neighborhood of $0$, it must have a local inverse that we can find as a power 
	series.
	It turns out that
	\[
		f(x,y) = (x, \sum_{n=0}^\infty x^nyx^n)
	\]
	is the formal power series representation of $\bbr^{-1}$.
	If one tries to write down a realization for $f$, then a contradiction is eventually attained showing that $f$ is not a rational power 
	series.
	Thus, $\bbr$ does not induce a rational automorphism.
	This doesn't give us too much information at the moment, since it reveals little about the injectivity of $\bbr$.
	
	Our alternative approach is to use the idea behind the conception of hyporationals: the matrix identity $\vecc(AXB) = \vecc(X)(A^T\otimes 
	B)$.
	Let $f = (\bbr^{-1})_2$.
	Since $f(x,y)$ satisfies the equation $f(x,y) = y + xf(x,y)x$, we evaluate on a pair of matrices $(X,Y)$ (when it makes sense) and note 
	we have
	\[
		f(X,Y) = Y + Xf(X,Y)X.
	\]
	Taking the vectorization of both sides and rearranging, we have
	\[
		\vecc(f(X,Y))(I_n\otimes I_n - X^T\otimes X) = \vecc(I_n)(I_n\otimes Y).
	\]
	Multiplying on the right by an inverse we see
	\[
		\vecc(f(X,Y)) = \vecc(I_n)(I_n\otimes Y)\left(I_n\otimes I_n - X^T\otimes X\right)^{-1}.
	\]
	Thus, for any matrix $X$ where $(I_n\otimes I_n - X^T\otimes X)$ is invertible we have $(X,Y)$ is in the domain of $\bbr^{-1}$.
	As pointed out in \ref{sec:TPFSF}, the function $(1\otimes 1-x'\otimes x)^{-1}\notin \C\skf{x'}\otimes \C\skf{x}$.
	However, a consequence of its (first) proof is that there is no free Euclidean open and Euclidean dense set upon which $(1\otimes 
	1-x'\otimes x)$ is invertible.
	Thus, $\bbr$ fails to satisfy requirement $(1)$ from the Conjecture and $\bbr$ does not induce a rational automorphism.
	
	To see why $q = (1\otimes 1-x'\otimes x)$ fails to be injective on a ``big" set, suppose $\Omega$ is any nonempty free open set upon 
	which $q$ is invertible.
	If $\lambda$ and $\mu$ are any eigenvalues of $X$ then $\lambda\mu$ is an eigenvalue of $X^T\otimes X$ and $1-\lambda\mu$ is an 
	eigenvalue of $q(X^T,X)$.
	Hence, if $X\in \Omega$, then for each eigenvalue $\lambda$ of $X$, $\lambda^{-1}$ is not an eigenvalue of any matrix in $\Omega$.
	Thus, the eigenvalues of $q(X^T,X)$, taken over all $X\in \Omega$ partition the complex plane.
	
	For any $X\in M_n(\C)$ let $\sigma(X)$ denote its set of eigenvalues and for any $U\subset M_n(\C)$ let $\sigma(U) = \cup_{X\in U} 
	\sigma(X)$.
	Since $\Omega$ is assumed to open, $\Omega[n]$ is open and $\sigma(\Omega[n])$ must contain an open set.
	Hence, if $\lambda\in \sigma(\Omega[n])$ is nonzero, then there exists an open set $W$ containing $\lambda^{-1}$ such that $W\cap 
	\sigma(\Omega[n]) = \varnothing$.
	
	However, $\sigma^{-1}(W)$ contains an open set of matrices, thus $\Omega$ cannot be free, open and dense.	
\end{example}


Condition $(1)$ is currently a best guess (the density was used to invoke complex analytic methods).
Understanding the domains of elements of $\C\skf{\bbx'\lra \bbx}$ seems to be quite important.









%############################################################################
%############################################################################
%############################################################################


\section{Rational Derivatives - \textit{SOLVED!}}
	\label{sec:RatDeriv}
	
Let $\bbx = \set{x_1,\dots, x_\ttg}$ and $\bbh = \set{h_1,\dots, h_\ttg}$ be a sets of freely noncommuting indeterminates.
Let $\C\fpsx$ denote the formal power series in $\bbx$ and let $\C\skf{\bbx}_0$ denote the rational formal power series.
If $S\in \C\fpsx$ and $w$ is a word, then $[S,w]$ is the coefficient of $w$ appearing in $S$.
For any word $w$ and series $S\in \C\fpsx$, we let $w^{-1}S = \sum_{v\in \fax} [S,wv]v$.


\begin{proposition}
	Suppose $f\in \C\fpsx$.
	If $Df(\x)[\h] \in \C\skf{\bbx,\bbh}_0$ then $f\in \C\skf{\bbx}_0$.
\end{proposition}

\begin{proof}
	For each $1\leq i\leq \ttg$ we let $\partial_i f := Df(\x)[0,\dots,0,h_i,0,\dots,0]$ and note $\partial_i f\in \C\skf{\bbx,\bbh}_0$.
	Next, we write
	\[
		f = c_0 + \sum_{i=1}^\ttg x_i f_i
	\]
	where each $f_i\in \C\fpsx$.
	Hence,
	\[
		\partial_i f = h_if_i + \sum_{j=1}^\ttg x_j \partial_i f_j\in \C\skf{\bbx,\bbh}_0
	\]
	and it follows that $h_i^{-1}\partial_if_i = f_i\in \C\skf{\bbx,\bbh}_0$ since $\partial_i f_i$ is contained in a stable submodule.
	Therefore, $f = c_0 + \sum_{i=1}^\ttg x_i f_i\in \C\skf{\bbx}_0$ since each $f_i$ is rational.
\end{proof}


The argument below shows it for generalized series. Notation and ideas are pulled from \cite{Vol18}.
Notably, if $v$ is a word, $L_v S = \sum_{w\in \fax} [S,vw]$.

\begin{remark}
	Let $\cA = M_n(\C)$ for some $n$.
	If $f\in \cA\fpsx$ then
	\[
		f = f_0 + \sum_{i=1}^\ttg L_{x_i} f
	\]
	and
	\[
		Df(\x)[\h] = \sum_{i=1}^\ttg L_{h_i}Df(\x)[\h] + \sum_{i=1}^\ttg L_{x_i}Df(\x)[\h].
	\]
	If $f_i(\x,h_i) = L_{h_i}Df(\x)[\h]$, then $f_i(\x,x_i) = L_{x_i}f$.
	Hence, if $Df(\x)[\h]$ is rational, then so is $L_{h_i}Df(\x)[\h]$ and consequently so is $L_{x_i}f$.
	Therefore, $f$ is rational since it is a sum of rational series.	
\end{remark}












%############################################################################
%############################################################################
%############################################################################




\section{Artin approximation theorem}
	\label{sec:ArtinApprox}


\begin{problem}
	Suppose $\bbk\fpsx$ is the algebra of formal power series with freely noncommuting indeterminates $\bbx = \set{x_1,\dots, x_\ttg}$ over 
	the field $\bbk$.
	Suppose $\bby = \set{y_1,\dots, y_\tth}$ is a distinct set of freely noncommuting indeterminates and let 
	\[
		f(\x,\y) = 0
	\]
	be a system of polynomial equations in $\bbk\fralg{\bbx,\bby}$, and let $c$ be a positive integer.
	Then given a formal power series solution $\hat{\y}(\x)\in \bbk\fpsx$, there is an algebraic solution $\y(\x)$ consisting of algebraic 
	power series such that
	\[
		\hat{\y}(\x) \equiv \y(\x) \mod (\x)^c
	\]
\end{problem}


In the classical setting, this problem uses the Implicit function theorem and some refinements of the IFT.




\section{Miscellaneous Problems without exposition}
	\label{sec:AlgMisc}

Max pencil kernels and Makar-Limanov

NC Noether Problem.

Artin-Schreier










\begin{thebibliography}{6}

\bibitem[Coh06]{Cohn06}
P.M. Cohn.
\newblock {\em Free Ideal Ring and Localizations in General Rings}.
\newblock Cambridge University Press, 2006.


\bibitem[PW95]{PW95}
H.J. Park and C.~Woodburn.
\newblock An algorithmic proof of {S}uslin's stability theorem for polynomial
  rings.
\newblock {\em Journal of Algebra}, 178(1):277 -- 298, 1995.

\bibitem[Umi07]{U07}
U.~U. Umirbaev.
\newblock The {A}nick automorphism of free associative algebras.
\newblock {\em J. Reine Angew. Math.}, 605:165--178, 2007.


\bibitem[Sch85]{Sch85}
Schofield, A. H.
\newblock Representations of rings over skew fields.
\newblock {\em London Mathematical Society Lecture Note Series}, 1985.


\bibitem[Swe70]{Swe70}
Moss~Eisenberg Sweedler.
\newblock A units theorem applied to {H}opf algebras and {A}mitsur cohomology.
\newblock {\em American Journal of Mathematics}, 92(1):259--271, 1970.



\bibitem[Vol18]{Vol18}
Jurij Vol{\v c}i{\v c}.
\newblock Matrix coefficient realization theory of noncommutative rational functions.
\newblock {\em Journal of Algebra}, 499: 397 - 437, 2018.

\bibitem[Vol19]{Vol19}
Jurij Vol{\v c}i{\v c}.
\newblock Free Bertini's Theorem and applications.
\newblock \url{https://arxiv.org/pdf/1908.08948.pdf}.


\end{thebibliography}









%############################################################################
%############################################################################
%############################################################################

%					CONVEXITY CHAPTER

%############################################################################
%############################################################################
%############################################################################

\chapter{Convexity Problems}



The following questions are due to Eric Evert (eric.evert@kuleuven.be) and Bill Helton (helton@math.ucsd.edu).
%\author[E. Evert]{Eric Evert${}^1$}
%\address{Eric Evert, Department of Mathematics\\
%  University of California \\
%  San Diego
%   }
%   \email{eevert@ucsd.edu}
%\thanks{${}^1$Research supported by the NSF grant
%DMS-1500835}
%\author[J.W. Helton]{J. William Helton${}^1$}
%\address{J. William Helton, Department of Mathematics\\
%  University of California \\
%  San Diego
%   }
%   \email{helton@math.ucsd.edu}


\section{Brief definitions}

We begin with basic definitions. Readers familiar with matrix convex sets and their extreme points may proceed directly to Section 
\ref{sec:MatExtreme}.

We warn that these questions and all claims regarding free spectrahedra restrict to the setting of real free spectrahedra. There can be 
important differences for free spectrahedra which contain tuples of complex self-adjoint matrices. 

\subsection{Free spectrahedra and matrix convex sets}

Let $SM_n (\R) ^g$ denote $g$-tuples of symmetric $n \times n$ matrices, and set $SM (\R)^g := \cup_n SM_n (\R)^g$. The following questions 
concern \textit{free spectrahedra}, i.e. solution sets to \textit{linear matrix inequalities}. Given a $g$-tuple $A \in SM_d (\R)^g$, a 
\df{linear matrix inequality} is an inequality of the form 
\[
L_A (x) = I + A_1 x_1 + \dots + A_g x_g \succeq 0.
\]
The map $L_A$ is called a \df{monic linear pencil}. For a matrix tuple $X = (X_1, \dots, X_g) \in SM(\R)^g$, the \df{evaluation} of $L_A$ on 
$X$ is 
\[
L_A (X) = I + A_1 \otimes X_1 + \dots + A_g \otimes X_g. 
\]
Here $\otimes$ denotes the Kronecker product. A \df{free spectrahedron} $\cD_A$ is the solution set 
\[
\cD_A = \{ X \in SM (\R)^g : L_A (X) \succeq 0 \}.
\]
We emphasize that a free spectrahedron contains matrix tuples of all sizes. We let $\cD_A (n)$ denote the set collection of $g$-tuples of $n 
\times n$ matrices in $\cD_A$. We say a free spectrahedron is \df{compact} if $\cD_A (n)$ is compact for each $n$.

Free spectrahedra are prototypical examples of \textit{matrix convex sets}. A \df{matrix convex combination} of a collection of tuples 
$\{Y^i\} \in SM(\R)^g$ is a finite sum of the form 
\beq
\label{eq:SumY}
\sum_{i=1}^k V_i^T Y^i V_i \qquad \mathrm{where} \qquad \sum_{i=1}^k V_i^T V_i = I_n.
\eeq
Here each $Y^i=(Y_1^i, \dots, Y_g^i)$ is a $g$-tuple of $n_i \times n_i$ matrices and the $V_i$ are $n_i \times n$ contractions. The product 
$V_i^T Y^i V_i$ is defined by 
\[
V_i^T Y^i V_i = (V_i^T Y_1^i V_i, \dots, V_i^T Y_1^i V_i).
\]
A matrix convex combination is said to be \df{proper} if each $V_i$ is surjective or \df{weakly proper} if $V_i \neq 0$ for each $i$. 

An important aspect of matrix convex combinations is that the $n_i$ need not be equal. I.e. the $Y^i$ may be tuples of matrices of different 
sizes. A set is called \df{matrix convex} if it is closed under matrix convex combinations. All free spectrahedra are matrix convex and every 
matrix convex set can be represented as a (perhaps infinite) intersection of free spectrahedra. 

A few warm-up exercises which are good for those interested in studying free spectrahedra and matrix convex sets are
\begin{enumerate}
\item Show every free spectrahedron is matrix convex.
\item Show that if $K$ is a matrix convex set, then $K(n)$ is convex in the classical sense.
\item Show that if $K$ is a matrix convex set and $K$ is nonempty, then $K$ contains matrix tuples of all sizes. In other words, show that if 
$K$ is matrix convex and there exists a tuple $X \in K(m)$ for some integer $m$, then for any integer $n$ there exists a tuple $Y \in K(m)$. 
\end{enumerate}

A (very) brief selection of recommended articles to read for basic information on Free spectrahedra and matrix convex sets is 
\cite{EW97,HM12,HKM13}.

\subsection{Extreme points of free spectrahedra}

There are several notions of an extreme point for a matrix convex set. The main types we consider are \textit{Euclidean (classical) extreme 
points, matrix extreme points}, and \textit{free extreme points}.



Suppose $\cD_A$ is a free spectrahedron. We say $X \in \cD_A (n)$ is a Euclidean extreme point of $\cD_A$ if $X$ is an extreme point of 
$\cD_A$ in the classical sense. 

A tuple  $X \in \cD_A(n)$  is a \df{matrix extreme point}
  of $\cD_A$
 if whenever it is represented as a proper matrix combination of the form
\eqref{eq:SumY} with $Y^i \in \cD_A(n_i)$,
then $n=n_i$ and for each $i$ there is a unitary $U_i$ so that  $X = U_i^T Y^i U_i$.

 A tuple  $X \in \cD_A(n)$  is an \df{free extreme point}   of $\cD_A$
 if whenever it is represented as a weakly proper matrix combination of the form \eqref{eq:SumY},  then
  for each $j$ either $n_j \leq n$ and there is a unitary $U_j$ so that  $X = U_j^T Y^j U_j$ (and hence $n_j=n$),
 or $n_j>n$ and there exists a $Z^j  \in \cD_A$ and a unitary $U_j$  such that $U_j^T Y^j U_j =  X\oplus Z^j$.
 
 In words, a tuple $X$ is a matrix extreme point a free spectrahedron if $X$ cannot be expressed as a nontrivial matrix convex combination of 
 elements of $\cD_A$ having size less than or equal to $X$, while a $X$ is a free extreme point if $X$ cannot be expressed as any nontrivial 
 convex combination of elements of the free spectrahedron. 
 
 We have the following relationship between extreme points of free spectrahedra.
 \begin{theorem}
 Let $\cD_A$ be a free spectrahedron. If $X$ is a free extreme point of $\cD_A$, then $X$ is a matrix extreme point of $\cD_A$. In addition, 
 if $X$ is a matrix extreme point of $\cD_A$, then $X$ is Euclidean extreme point of $\cD_A$. 
 \end{theorem}
 
 A (very) brief selection of recommend articles to read for an introduction to extreme points of free spectrahedra and matrix convex sets is 
 \cite{WW99,EHKM18,EH19}.
 
It is worth mentioning that free extreme points can be seen as a finite dimensional version of the classical dilation theoretic Arveson 
boundary. We will not discuss this perspective further; however we recommend \cite{A69,DK15,DK+} as a starting ponit for the interested 
reader. Also see \cite{F00,F04} for discussions of matrix extreme points from this perspective.
 
 We now present a list of questions related to free spectrahedra and their extreme points.

\section{Are matrix extreme points equal to free extreme points in a free spectrahedron}
\label{sec:MatExtreme}

While it is known that the set of matrix extreme points can be a proper subset of the Euclidean extreme points of a free spectrahedron, there 
are no known examples of a matrix extreme point which is not a free extreme point of a free spectrahedron.

\begin{question}
\label{question:MissingMatExtreme}
Let $\cD_A$ be a free spectrahedron with $A$ a tuple of real symmetric matrices. Is the set of matrix extreme points of $\cD_A$ equal to the 
set of free extreme points of $\cD_A$. 
\end{question}

For general matrix convex sets there are examples of matrix extreme points which are not free extreme points. More dramatically, there are 
examples of matrix convex sets which are not free spectrahedra which have no free extreme points, e.g. see \cite{E18}. On the other hand, 
\cite{WW99} shows that all (compact) matrix convex sets are spanned by matrix extreme points. It is known that all free spectrahedra are 
spanned by free extreme points, see \cite[Theorem 1.1]{EH19}.

\section{Noncommutative varieties}

A small number of free spectrahedra are known to have the property that their free extreme points are determined by a noncommutative variety. 
Defined by example, a \df{noncommutative (NC) polynomial} is a polynomial in the noncommuting indeterminates $x=(x_1,\dots,x_g)$, e.g. 
\[
p(x) = x_1 x_2 + 2 x_3 x_1 x_2 x_1
\]
is a NC polynomial. \df{Evaluation} of a NC polynomial on a $g$-tuple of matrices is defined by replacing $x_i \to X_i$, e.g. 
\[
p(X) = X_1 X_2 + 2 X_3 X_1 X_2 X_1. 
\]
A noncommutative variety is the zero set of a finite collection of NC polynomials. That is, given a finite collection $\{p_i\}_{i=1}^k$ of NC 
polynomials an \df{noncommutative variety}\footnote{This terminology varies from paper to paper.} is a set of the form 
\[
\{ X \in SM (\R)^g : p_i(X) = 0 \ \mathrm{for \ all \ } i=1,\dots,k\}.
\]

\begin{question}
Which free spectrahedra have the property that the set of free extreme points of the free spectrahedron is given by a noncommutative variety? 
Are there examples of free spectrahedra which do not have this property.
\end{question}

The free spectrahedra known to have free extreme points determined by a noncommutative variety are
\begin{enumerate}
\item The free cube in $g$-variables. I.e. the free spectrahedron $\mathcal{C}^g$ defined by
\[
\mathcal{C}^g = \{X \in SM(\R)^g : X_i^2 \preceq I \ \mathrm{for \ all} \ i = 1,\dots, g\}.
\]
\item Free simplices. I.e. free spectrahedra in $g$ variables defined by a tuple of $g+1 \times g+1$ diagonal matrices.
\item Free quadrilaterals. I.e. free spectrahedra in $2$ variables defined by a tuple of $4 \times 4$ diagonal matrices.
\item The spin ball. I.e. the free spectrahedron with defining pencil $L_A$ where 
\[
A = \left(\begin{pmatrix}
1 & 0 \\ 
0 & -1
\end{pmatrix},
\begin{pmatrix}
0 & 1 \\ 
1 & 0
\end{pmatrix}
\right)
\]
\end{enumerate}

Limited numerical evidence suggests that free spectrahedra with free extreme points determined by a noncommutative variety are rare. See 
\cite{E+} for details.



\section{Free spectrahedrops}

We now give various questions related to free spectrahedrops, i.e. projections of free spectrahedra; some call these spectrahedral shadows. 
Let $\cD_{(A,\tilde{A})}$ be a free spectrahedron with elements $(X,Y)$. That is $\cD_{(A,\tilde{A})}$ is the set of solutions to the linear 
matrix inequality
\[
I+A_1 \otimes X_1 + \cdots + A_g \otimes X_g + \tilde{A}_1 \otimes \tilde{Y}_1 + \cdots \tilde{A}_{\tilde{g}} \otimes \tilde{Y}_{\tilde{g}} 
\succeq 0. 
\]
We define the \df{free spectrahedrop} $P_X \cD_{(A,\tilde{A})}$ by
\[
P_X \cD_{(A,\tilde{A})} = \{ X \in SM(\R)^g : \mathrm{\ There \ exists\ a \ } Y \in SM(\R)^{\tilde{g}} \ \mathrm{such\ that\ } (X,Y) \in 
\cD_{(A,\tilde{A})} \}
\]
See \cite{HKMjems} for details about free spectrahedrops. 
 
\subsection{Extreme points of free spectrahedrops}

\begin{question}
Let $P_X \cD_{(A,\tilde{A})}$ be a free spectrahedrop which is closed under complex conjugation. Is the set of free extreme points of $P_X 
\cD_{(A,\tilde{A})}$ nonempty? Moreover, is $P_X \cD_{(A,\tilde{A})}$ equal to the matrix convex hull of its free extreme points?
\end{question}

\cite{EHKM18} shows that the set of free extreme points of a free spectrahedron is nonempty, and \cite{EH19} shows that every free 
spectrahedron is the matrix convex hull of its free extreme points. 

\begin{question}
Let $K$ be a free spectrahedrop, and let $\cD_{(A,\tilde{A})}$ be a free spectrahedron such that $P_X \cD_A = K$. Suppose $Y$ is a free 
extreme point of $K$. Is there always a tuple $\tilde{Y}$ such that $(Y,\tilde{Y})$ is a free extreme point of $\cD_A$. 
\end{question}

Note: A negative answer to the above question would resolve Question \ref{question:MissingMatExtreme}. Using the fact that a matrix convex 
set at level $n$ is spanned by its matrix extreme points at level $n$, one show that there is always a $\tilde{Y}$ such that $(Y,\tilde{Y})$ 
is matrix extreme.

\subsubsection{Drops and hulls}

\begin{question}
Characterize which free matrix convex sets are free spectrahedrops. More generally, characterize free sets which are projections of free 
semialgebraic sets: This is one of the most basic problems in free real algebraic geometric.
\end{question}

\cite{HM12} shows that if $K$ is free semialgebraic set which is level wise convex set and has $0$ in its interior, then $K$ is a free 
spectrahedron. It would be very interesting to have some kind of characterization of free spectrahedrops. 

Before stating the next question we introduce a small amount of notation. Given a NC polynomial $p$ with $p(0)=1$ we let $\cD_p$ denote the 
connected component around $0$ of $\{X : p (X) \succeq 0\}$. 

\begin{question}
For $p(x) = 1-x_1^2-x_2^4$ is the matrix convex hull of $\cD_p$ a free spectrahedrop. In general, for which NC polynomials $p$ is the matrix 
convex hull of $\cD_p$ a free spectrahedrop. 
\end{question}

$\cD_p$ is often called the bent TV screen. See \cite[Section 7.3]{EHKM18} for an in depth discussion of this domain. The only known examples 
for the above question are sets which are either convex or ``secretly" convex, see \cite{HKM16}.



\section{Free polar duals}

Given a matrix convex set $K \subset SM(\R)^g$, the \df{polar dual} of $K$, denoted $K^\circ$ is the set
\[
K^\circ = \{ Y \in SM(\R)^g : L_X (Y) \succeq 0 \mathrm{\ for\ all\ } X \in K\}. 
\]

\begin{question}
Is there a free spectrahedron $\cD_A$ such that the polar dual of $\cD_A$ is also a free spectrahedron and such that $\cD_A$ is not a free 
simplex. Equivalently, if $\cD_A$ is a free spectrahedron which is not a simplex, does $\cD_A$ necessarily have infinitely many (unitary 
equivalence classes of) free extreme points.
\end{question}

\cite[Corollary 6.8]{EHKM18} and \cite[Theorem 4.7]{FNT17} independently show that the polar dual of a free simplex is again a free simplex. 
In \cite{EHKM18} it is shown that the polar dual of a matrix convex set is a free spectrahedron if and only if the matrix convex set has 
finitely many (unitary equivalence classes of) free extreme points. It is not known if there are free spectrahedra which are not free 
simplices that have finitely many free extreme points. 


\begin{thebibliography}{1}
	\bibitem[A69]{A69} W. Arveson:
	{\it Subalgebras of $C^*$-algebras}, Acta Math. {\bf 123} (1969) 141-224.
		
	
	\bibitem[DK15]{DK15} K.R. Davidson, M. Kennedy:
	{\it The Choquet boundary of an operator system}, Duke Math. J. {\bf 164} (2015) 2989-3004.
	
	\bibitem[DK+]{DK+}
	K.R. Davidson, M. Kennedy:
	\textit{Noncommutative Choquet Theory},
	preprint \url{https://arxiv.org/pdf/1905.08436.pdf}.
	
	
	\bibitem[EW97]{EW97} E.G. Effros, S. Winkler: {\it Matrix convexity: operator analogues of the bipolar and Hahn-Banach theorems}, J. 
	Funct. Anal. {\bf 144} (1997) 117-152.
	
	\bibitem[E18]{E18} E. Evert: {\it Matrix convex sets without absolute extreme points}, Linear Algebra Appl. {\bf 537} (2018) 
	287-301.		
	\bibitem[E+]{E+}
	E. Evert: {\it The Arveson boundary of a Free Quadrilateral is given by a noncommutative variety}, preprint
		https://arxiv.org/abs/2008.13250
	
	\bibitem[EH19]{EH19} E. Evert, J.W. Helton: {\it Arveson extreme points span free spectrahedra},	Math. Ann. {\bf 375} (2019) 
	629-653.
	
	\bibitem[EHKM18]{EHKM18}
	E. Evert, J.W. Helton, I. Klep, S. McCullough:
	{\it Extreme points of matrix convex sets, free spectrahedra and dilation theory},
	J. of Geom. Anal. {\bf 28} (2018) 1373-1498.
	
	\bibitem[EOYH19]{EOYH19} E. Evert, M. de Oliveira, J. Yin, and J.W. Helton: {\it NCSE 1.0: An NCAlgebra for optimization over free 
	spectrahedra}, Available online, Jan. 2019. URL:
	https://github.com/NCAlgebra/UserNCNotebooks/tree/master/NCSpectrahedronExtreme
	
	\bibitem[F00]{F00} D.R. Farenick:
	{\it Extremal matrix states on operator systems}, J. London Math. Soc. {\bf 61} (2000) 885-892.
	
	\bibitem[F04]{F04} D.R. Farenick:
	{\it Pure matrix states on operator systems}, Linear Algebra Appl. {\bf 393} (2004) 149-173.
	
		
	\bibitem[FNT17]{FNT17} T. Fritz, T. Netzer, A. Thom: {\it Spectrahedral Containment and Operator Systems with Finite-dimensional 
	Realization}, SIAM J. Appl. Algebra Geom. {\bf 1} (2017) 556-574.
	
	
	\bibitem[HKM13]{HKM13} J.W. Helton, I. Klep, S. McCullough: {\it The matricial relaxation of a linear matrix inequality}, Math. 
	Program. {\bf 138} (2013) 401-445.
	
	\bibitem[HKM16]{HKM16}
	J.W. Helton, I. Klep, S. McCullough:
	{\it Matrix convex hulls of free semialgebraic sets}, Trans. Amer. Math. Soc. {\bf 368} (2016) 3105--3139.
	
	
	\bibitem[HKM17]{HKMjems}
	J.W. Helton, I. Klep, S. McCullough:
	{\it The tracial Hahn-Banach theorem, polar duals, matrix convex sets, and projections of free spectrahedra}, J. Eur. Math. Soc. {\bf 6} 
	(2017) 1845--1897.
	
	
	\bibitem[HM12]{HM12}
	J.W. Helton, S. McCullough: {\it Every free basic convex semi-algebraic set has an LMI representation}, Ann. of Math. (2) {\bf 176} 
	(2012) 979-1013.
	
	
	\bibitem[WW99]{WW99}
	C. Webster and S. Winkler:
	{\it The Krein-Milman Theorem in Operator Convexity},  Trans Amer. Math. Soc. {\bf 351} (1999) 307-322.

\end{thebibliography}

%\begin{itemize}
%\item What sets spectradrops apart from general matrix convex sets?
%
%\item Positivstellensatz for spectradrops
%
%\item Including a drop into the matrix cube, or other spectrahedron. A variation
%of asking about approximating above by a specrahedron.
%
%\end{itemize}




%############################################################################
%############################################################################
%############################################################################

%					GEOMETRY CHAPTER

%############################################################################
%############################################################################
%############################################################################

\chapter{Geometric Problems}




\section{Injektivit{\"a}tstellensatz? COUNTEREXAMPLE found!}

Suppose $\bbp\in (\C\fralg{\bbx})^\tth$, that is $\bbp = (p_1,p_2,\dots, p_\tth)$.
If $q$ is a free polynomial that is in the unital algebra generated by $\bbp$, then there exists some free polynomial $f$ (in 
$\tth$-variables) such that $q = f\circ \bbp$.
It is quickly seen that in this case $q$ inherits the injectivity failings of $\bbp$.
That is, $q(X) = q(Y)$ whenever $\bbp(X) = \bbp(Y)$.

The question is whether the above condition is a certificate of inclusion in the algebra generated by $\bbp$.

\begin{problem}
	Suppose $\bbp\in (\C\fralg{\bbx})^\tth$ and $q\in \C\fralg{\bbx}$.
	If $q(X) = q(Y)$ whenever $\bbp(X) = \bbp(Y)$ (for $X,Y\in M_n(\C)^\ttg$ or perhaps a different free set), then is it the case that $q$ 
	is in the unital algebra generated by $\bbp$?
\end{problem}

The answer is no! Let $\bbp(x,y) = (x, xy, yx, y + yxy)$.
Note that if $D\bbp(X,Y)[H,K] = 0$, then $H = 0$, and consequently, $K=0$, thus $\bbp$ is injective globally.
However, $y$ is not in the algebra generated by $\bbp$.

The explanation is that the algebra homomorphism is not surjective, but rather, epic.


%Since we are examining the unital algebra, it is safe to assume that $\bbp(0) = 0$ and $q(0) = 0$.
%With this additional assumption, we have in particular that $q(X) = 0$ whenever $\bbp(X) = 0$.
%Hence, $q$ is in the ideal generated by $\bbp$ (however this is naturally the case when $q$ is in the unital algebra generated by $\bbp$ and 
%they both send $0$ to $0$).
%However, membership in the ideal is not enough to guarantee that a polynomial shares the same lack of injectivity.
%For example, $x^3$ is in the ideal generated by $x^2$, but $(-1)^2 = 1 = 1^2$ while $(-1)^3 \neq (1)^3$.
%
%
%We have the following result for free functions that gives a connection between our problem's conjectured certificate and zero sets of 
%derivatives.

%\begin{proposition}
%	Suppose $\bm{F}$ is a free mapping (a $d$-tuple of free maps) on a free domain $\Omega$ and $G$ is a free map.
%	Then $G(X) = G(Y)$ whenever $\bm{F}(X) = \bm{F}(Y)$ (for $X,Y\in \Omega$) if and only if $DG(Z)[W] = 0$ whenever 
%	$D\bm{F}(Z)[W] = 0$ (for $Z\in \Omega$ and $W\in M(\C)^\ttg$).
%\end{proposition}
%
%Thus, some mileage may be gained from investigating the zero sets of derivatives.


\section{Polynomial Automorphisms between Free Loci}

Suppose $A = (A_1,\dots, A_\ttg)\in M_k(\C)^\ttg$ and let $L_A(x) = I_k - \sum_{i=1}^\ttg A_ix_i$.
The function $L_A$ is a monic linear pencil and it can be evaluated on $\ttg$-tuples of matrices in the natural way.
If $X = (X_1,\dots, X_\ttg)\in M_n(\C)^\ttg$ then
\[
	L_A(X) = I_k\otimes I_n - \sum_{i=1}^\ttg A_i\otimes X_i
\]
where $\otimes$ is the Kronecker product. For each $n\in \Z^+$ define
\[
	\mathscr{Z}_n(L_A) = \set{X\in M_n(\C)^\ttg \, : \, \det(L_A(X)) = 0}
\]
and let
\[
	\mathscr{Z}_n(L_A) = \bigsqcup_{n=1}^\infty \mathscr{Z}_n(L_A).
\]
That is, $\mathscr{Z}(L_A)$ is the free set whose levels are $\mathscr{Z}_n(L_A)$ and we call $\mathscr{Z}(L_A)$ the \df{free locus} of $L_A$.
There are many significant results involving free loci \cite{KlVol17}.

\begin{problem}
	Suppose $A\in M_n(\C)^\ttg$ and $B\in M_m(\C)^\ttg$. Describe the polynomial automorphisms between $\mathscr{Z}(L_A)$ and 
	$\mathscr{Z}(L_B)$. When non-trivial automorphisms exist, how are $A$ and $B$ related? If needed, assume $n=m$.
\end{problem}




\begin{thebibliography}{1}

\bibitem[KlVol17]{KlVol17}
Klep, I. and Vol{\v c}i{\v c}, J.
\newblock Free loci of matrix pencils and domains of noncommutative rational functions.
\newblock {\em Commentarii Mathematici Helvetici}, 92(1):105--130, 2017.



\end{thebibliography}




















\chapter{Analysis}


\section{When can the Free Implicit Function Theorem be applied?}
In order to apply the Free Implicit Function Theorem to a function $f$, there must be a point $X$ so that the derivative $Df(X)$ has full rank.
In contrast to the commutative case, there are functions so that the Free Implicit Function Theorem cannot be applied at any point.
Specifically, the commutator $[x,y] = xy - yx$ cannot have the Free Implicit Function Theorem applied to it at any point.

\begin{conjecture}
	Suppose $f$ is a free analytic function.
	If $Df(X)$ is not full rank for all $X$ in a free domain, then $f$ is built out of commutators.
\end{conjecture}

%\begin{itemize}
%	\item Local Operator Monotonicity vs. Global Operator Monotonicity
%	
%%	\item Jerk functions and their realizations
%	
%	\item Weierstrass preparation theorem
%	
%	\item NC rational function defined on $\B(\cH)$
%
%\end{itemize}


\section{What functions are bounded on $B_{\delta}$'s?}

Suppose $\delta$ is an $I\times J$ matrix of free polynomials.
We define
\[
	B_\delta = \set{X\in M(\C)^\ttg \, :\, \norm{\delta(X)}<1}.
\]
In some cases, $B_\delta$ is empty at each level: $\norm{1 - XY + YX}\geq 1$ for all matrix evaluations.
However, $B_{\delta/r}$ will be nonempty for some $r>0$.

\begin{problem}
	Suppose $\delta$ is an $I\times J$ matrix of free polynomials and $q$ is a free map.
	When is $q$ bounded on $B_\delta$? If $q$ is a free polynomial, when is $q$ bounded on $B_{\delta/r}$ for all $r>0$?
\end{problem}

Another related problem is the following {\color{red} This problem has a counterexample, explained below!}

\begin{problem}
	Suppose $\delta$ is an $I\times J$ matrix of free polynomials.\
	If there exists $\ep>0$ such that $\delta$ is injective on $B_{\delta/1+\ep}$, then is $B_\delta$ bounded?
\end{problem}

The above problem is false if we do not include the $1+\ep$. Let $\delta = (x, y(1-x))$.
If $\norm{\delta(X,Y)} = \norm{(X,Y(I-X))}<1$ then certainly $\norm{X}<1$ so $\delta$ is injective on $B_\delta$.
However, $B_\delta$ is not bounded.

In fact, if $\delta(x,y) = \frac{1}{4}(x, xy, yx, y - yxy)$, then $\delta$ is globally injective.
However, $\delta(t^{-1}, t) = \frac{1}{4}(t^{-1}, 1, 1, 0)$ has norm less than $1$ when $t>1$, implying that $B_\delta$ is unbounded even on 
the scalars!













\end{document}












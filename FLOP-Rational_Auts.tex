
\begin{problem}
	Suppose $\bbr\in (\C\skf{x_1,\dots, x_\ttg})^\ttg$. Find an evaluation criterion that is necessary and sufficient for the induced map $\rho:\C\skf{x_1,\dots, x_\ttg}\to\C\skf{x_1,\dots, x_\ttg}$ ($\rho(x_i) = \bbr_i$) to be an automorphism.
\end{problem}

An evaluation criterion is some condition on $\bbr$ when we treat it as a function.
For example, injective, surjective, etc.

\begin{conjecture}
	\label{conj:rats auts conj}
	Suppose $\bbr\in (\C\skf{x_1,\dots, x_\ttg})^\ttg$.
	The following are equivalent:
	\begin{enumerate}
		\item there exists a free, Euclidean open and Euclidean dense set $\Omega$ such that $\bbr\lvert_\Omega$ is injective;
		\item $J_\bbr$ is an invertible element of $M_\ttg(\C\skf{\bbx'}^{opp}\otimes \C\skf{\bbx})$;
		\item the induced map $\rho:\C\skf{\bbx}\to \C\skf{\bbx}$ given by $\rho(x_i) = \bbr_i$ is an automorphism.
	\end{enumerate}
\end{conjecture}


This is simply a best guess conjecture at the moment. Clearly, $(3)\Rightarrow (1),(2)$.
Condition $(1)$ is seemingly strange, but the na{\"i}ve attempt of requiring injective on its domain is insufficient.

\begin{example}
	Let $\bbr(x,y) = (x, y - x^2y)$.
	This induces a rational automorphism, however $\bbr(1,\alpha) = \bbr(1,\beta)$, hence $\bbr$ is not injective on its domain (in fact, the points where it is not injective are exactly the points where $\bbr^{-1}$ is not defined).
	If $\Omega = \set{(X,Y)\in M_n(\C)^2 \, : \, \det(I_n - X^2)\neq 0}$, then $\Omega$ is free, Euclidean open and dense and $\bbr$ is injective on $\Omega$.
\end{example}

Let us see a slightly harder example.

\begin{example}
	Let $\bbr(x,y) = (x, y - xyx)$.
	Naturally $\bbr$ is not injective on $\C^2$ since $\bbr(1,\alpha) = \bbr(1,\beta)$.
	Moreover, $\bbr$ does not induce a rational automorphism and this fact is a bit harder to see.
	
	The first observed reason why is found by using formal power series.
	Since the derivative of $\bbr$ is invertible on some free neighborhood of $0$, it must have a local inverse that we can find as a power series.
	It turns out that
	\[
		f(x,y) = (x, \sum_{n=0}^\infty x^nyx^n)
	\]
	is the formal power series representation of $\bbr^{-1}$.
	If one tries to write down a realization for $f$, then a contradiction is eventually attained showing that $f$ is not a rational power series.
	Thus, $\bbr$ does not induce a rational automorphism.
	This doesn't give us too much information at the moment, since it reveals little about the injectivity of $\bbr$.
	
	Our alternative approach is to use the idea behind the conception of hyporationals: the matrix identity $\vecc(AXB) = \vecc(X)(A^T\otimes B)$.
	Let $f = (\bbr^{-1})_2$.
	Since $f(x,y)$ satisfies the equation $f(x,y) = y + xf(x,y)x$, we evaluate on a pair of matrices $(X,Y)$ (when it makes sense) and note we have
	\[
		f(X,Y) = Y + Xf(X,Y)X.
	\]
	Taking the vectorization of both sides and rearranging, we have
	\[
		\vecc(f(X,Y))(I_n\otimes I_n - X^T\otimes X) = \vecc(I_n)(I_n\otimes Y).
	\]
	Multiplying on the right by an inverse we see
	\[
		\vecc(f(X,Y)) = \vecc(I_n)(I_n\otimes Y)\left(I_n\otimes I_n - X^T\otimes X\right)^{-1}.
	\]
	Thus, for any matrix $X$ where $(I_n\otimes I_n - X^T\otimes X)$ is invertible we have $(X,Y)$ is in the domain of $\bbr^{-1}$.
	As pointed out in \ref{sec:TPFSF}, the function $(1\otimes 1-x'\otimes x)^{-1}\notin \C\skf{x'}\otimes \C\skf{x}$.
	However, a consequence of its (first) proof is that there is no free Euclidean open and Euclidean dense set upon which $(1\otimes 1-x'\otimes x)$ is invertible.
	Thus, $\bbr$ fails to satisfy requirement $(1)$ from the Conjecture and $\bbr$ does not induce a rational automorphism.
	
	To see why $q = (1\otimes 1-x'\otimes x)$ fails to be injective on a ``big" set, suppose $\Omega$ is any nonempty free open set upon which $q$ is invertible.
	If $\lambda$ and $\mu$ are any eigenvalues of $X$ then $\lambda\mu$ is an eigenvalue of $X^T\otimes X$ and $1-\lambda\mu$ is an eigenvalue of $q(X^T,X)$.
	Hence, if $X\in \Omega$, then for each eigenvalue $\lambda$ of $X$, $\lambda^{-1}$ is not an eigenvalue of any matrix in $\Omega$.
	Thus, the eigenvalues of $q(X^T,X)$, taken over all $X\in \Omega$ partition the complex plane.
	
	For any $X\in M_n(\C)$ let $\sigma(X)$ denote its set of eigenvalues and for any $U\subset M_n(\C)$ let $\sigma(U) = \cup_{X\in U} \sigma(X)$.
	Since $\Omega$ is assumed to open, $\Omega[n]$ is open and $\sigma(\Omega[n])$ must contain an open set.
	Hence, if $\lambda\in \sigma(\Omega[n])$ is nonzero, then there exists an open set $W$ containing $\lambda^{-1}$ such that $W\cap \sigma(\Omega[n]) = \varnothing$.
	
	However, $\sigma^{-1}(W)$ contains an open set of matrices, thus $\Omega$ cannot be free, open and dense.	
\end{example}


Condition $(1)$ is currently a best guess (the density was used to invoke complex analytic methods).
Understanding the domains of elements of $\C\skf{\bbx'\lra \bbx}$ seems to be quite important.








%\begingroup
%\renewcommand{\addcontentsline}[3]{}% Remove functionality of \addcontentsline
%\renewcommand{\section}[2]{}% Remove functionality of \section
%
%\vspace{1em}
%
%\begin{center}
%	{\normalsize \textsc{References}}
%\end{center}
%
%
%\begin{thebibliography}{Swe70}
%
%%\bibitem[Coh06]{Cohn06}
%%P.M. Cohn.
%%\newblock {\em Free Ideal Ring and Localizations in General Rings}.
%%\newblock Cambridge University Press, 2006.
%%
%%\bibitem[PW95]{PW95}
%%H.J. Park and C.~Woodburn.
%%\newblock An algorithmic proof of {S}uslin's stability theorem for polynomial
%%  rings.
%%\newblock {\em Journal of Algebra}, 178(1):277 -- 298, 1995.
%
%\bibitem[Swe70]{Swe70}
%Moss~Eisenberg Sweedler.
%\newblock A units theorem applied to hopf algebras and amitsur cohomology.
%\newblock {\em American Journal of Mathematics}, 92(1):259--271, 1970.
%
%\end{thebibliography}
%
%
%\endgroup
An automorphism $\tau$ of the free algebra $\C\fralg{x_1,\dots, x_\ttg}$ is {\bf elementary} if $\tau:x_i \mapsto x_i$ 
	for $i\neq j$ and $\tau:x_j \mapsto cx_j + f$, where $c\in \C$ and $f\in \C\fralg{x_1,\dots, x_{j-1}, x_{j+1},\dots, x_\ttg}$.
We say an automorphism is {\bf tame} if it is a composition of elementary automorphisms.
If the automorphism is not tame, then it is {\bf wild}.

In \cite{U07} it is shown that the Anick automorphism, 
\[
	\delta(x,y,z) = (x + z(xz - zy), y + (xz - zy)z, z)\in \C\fralg{x,y,z}
\]
is wild.
Looking at it, the Anick automorphism doesn't seem particularly ``wild," so maybe this can be improved?
Perhaps it is something about Jacobian matrices?

\begin{definition}
	We say an automorphism $\tau$ of $\C\fralg{x_1,\dots, x_\ttg}$ is \textbf{Jacobian Tame} if
	$J_\tau = DE_1\dots E_k$, where $D\in M_\ttg(\C\fralg{\bbx'}^{opp}\otimes \C\fralg{\bbx})$ is diagonal and
	$E_1,\dots, E_k\in M_\ttg(\C\fralg{\bbx'}^{opp}\otimes \C\fralg{\bbx})$ are elementary matrices.
	
	If such a factorization does not exist, then we say $\tau$ is \textbf{Jacobian wild}.
\end{definition}


\begin{problem}
	Are there any Jacobian wild automorphisms of the free algebra?
	The only known (to me) example of a wild automorphism of $\C\fralg{x,y,z}$ is Jacobian tame (this is explained below).
	
	In the commutative case there are no Jacobian wild automorphisms.
	Every automorphism of $\C[t_1,t_2]$ is tame, hence Jacobian tame.
	If $\phi$ is an automorphism of $\C[t_1,\dots, t_\ttg]$ with $\ttg>2$, then its Jacobian matrix
	$J_\phi\in M_\ttg(\C[t_1,\dots, t_\ttg,t_1',\dots,t_\ttg'])$ factors into such a product by Suslin's Stability Theorem (see \ref{sec:Elem_Mats}), thus is Jacobian tame.
\end{problem}

I will be using the transposed Jacobian matrix: $J_\tau \in M_\ttg(\C\fralg{\bbx'}^{opp}\otimes \C\fralg{\bbx})$ where the $i^\text{th}$ column of $J_\tau$ corresponds to the derivatives of $\tau_i$.

It turns out that the Anick automorphism 
\[
	\delta(x,y,z) = (x + z(xz - zy), y + (xz - zy)z, z)\in \C\fralg{x,y,z}
\]
has a Jacobian matrix that can be written as a product of elementary matrices.
The Jacobian matrix of $\delta$ is
\[
	J_\delta = 
	\bpm
		1\otimes 1 + z'\otimes z & 1\otimes z^2 & 0 \\
		-(z')^2\otimes 1 & 1 - z'\otimes z & 0 \\
		\zeta_1 & \zeta_2 & 1
	\epm
\]
where $\zeta_1 = 1\otimes xz+ x'z'\otimes 1 - 1\otimes zy - z'\otimes y$ and 
	$\zeta_2 = x\otimes z + z'x'\otimes 1 - 1\otimes yz - y'z'\otimes 1$.
Let $E_{i,j}(\alpha) = I + \alpha e_{i,j}$ ($e_{i,j}$ has a $1$ in the $i,j$ entry and 0's elsewhere) and observe
\[
	J_\delta E_{3,1}(-\zeta_1)E_{3,2}(-\zeta_2) = 
	\bpm
		1\otimes 1 + z'\otimes z & 1\otimes z^2 & 0 \\
		-(z')^2\otimes 1 & 1 - z'\otimes z & 0 \\
		0 & 0 & 1
	\epm.
\]
In 1966, Cohn proved that
\[
	\mathfrak{C} = 
	\bpm
		1\otimes 1 + z'\otimes z & 1\otimes z^2\\
		-(z')^2\otimes 1 & 1 - z'\otimes z
	\epm
\]
cannot be written as a product of elementary matrices over $\mathbb{F}[z'\otimes 1,1\otimes z]$.
This is a crucial aspect of Umirbaev's (\cite{U07}) proof that the Anick automorphism is wild.
However, Park and Woodburn \cite{PW95} give a decomposition of $(\begin{smallmatrix} \mathfrak{C} & 0 \\ 0 & 1 \end{smallmatrix})$ 
	into a product of elementary matrices:
\begin{align*}
	J_\delta &E_{3,1}(-\zeta_1)E_{3,2}(-\zeta_2) =
	\bpm
		1\otimes 1 + z'\otimes z & 1\otimes z^2 & 0 \\
		-(z')^2\otimes 1 & 1 - z'\otimes z & 0 \\
		0 & 0 & 1
	\epm\\
	= &	
	E_{2,3}(-z'\otimes 1)E_{1,3}(1\otimes z)E_{3,2}(1\otimes z)E_{3,1}(z'\otimes 1)\\
	& E_{2,3}(z'\otimes 1)E_{1,3}(-1\otimes z)E_{3,2}(-1\otimes z)E_{3,1}(-z'\otimes 1).
\end{align*}
Thus
\begin{align*}
	J_\delta = & E_{2,3}(-z'\otimes 1)E_{1,3}(1\otimes z)E_{3,2}(1\otimes z)E_{3,1}(z'\otimes 1)\\
	&E_{2,3}(z'\otimes 1)E_{1,3}(-1\otimes z)E_{3,2}(-1\otimes z)E_{3,1}(-z'\otimes 1)E_{3,2}(\zeta_2)E_{3,1}(\zeta_1).
\end{align*}
Thus, the Jacobian matrix of $\delta$ factors as a product of elementary matrices even though it is wild.


\vspace{1em}



Naturally, a positive answer to the Problem in \ref{sec:Elem_Mats} would show that every automorphism is Jacobian Tame.
On the other hand, a negative answer to the Problem in \ref{sec:Elem_Mats} would only serve to complicate things since a Jacobian matrix will certainly look quite different from a generic matrix in $\GL_\ttg(\C\fralg{\bbx'}^{opp}\otimes \C\fralg{\bbx})$.



%\begingroup
%\renewcommand{\addcontentsline}[3]{}% Remove functionality of \addcontentsline
%\renewcommand{\section}[2]{}% Remove functionality of \section
%
%\vspace{1em}
%
%\begin{center}
%	{\normalsize \textsc{References}}
%\end{center}
%
%
%\begin{thebibliography}{Umi07}
%
%\bibitem[PW95]{PW95}
%H.J. Park and C.~Woodburn.
%\newblock An algorithmic proof of {S}uslin's stability theorem for polynomial
%  rings.
%\newblock {\em Journal of Algebra}, 178(1):277 -- 298, 1995.
%
%\bibitem[Umi07]{U07}
%U.~U. Umirbaev.
%\newblock The {A}nick automorphism of free associative algebras.
%\newblock {\em J. Reine Angew. Math.}, 605:165--178, 2007.
%
%\end{thebibliography}
%
%
%\endgroup





